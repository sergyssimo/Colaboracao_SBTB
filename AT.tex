\chapter{Introdução}

Prezados irmãos,

Graça e paz!

Novamente coloco à disposição de todos, para análise e consideração,
mais esta contribuição ao nosso estimado texto bíblico traduzido e
publicado pela Sociedade Bíblica Trinitariana do Brasil, também
conhecido como Almeida Corrigida e Fiel. Desta feita, faço colocações
pertinentes ao Pentateuco e aos Livros Históricos. Pretendo, o quanto
me for possível, até ao fim do ano, disponibilizar o restante (livros
Sapienciais e Proféticos), concluindo, desta forma, todo o Antigo
Testamento e, com isso, a Bíblia como um todo, já que há alguns meses
já havia disponibilizado aos irmãos observações e sugestões ao Novo
Testamento.

Como da outra vez, são observações e correções de caráter
quase que exclusivamente gramaticais, mas também outras dúvidas e
sugestões dizem respeito à tradução em si de determinados trechos ou
termos usados.

Ainda gostaria de salientar que, apesar de citar passagens de outras
versões, como a King James e a Edição Contemporânea, não compartilho
da opinião de que se deve traduzir a partir da renomada King James ou de
qualquer outra versão, mas única e exclusivamente do Texto Massorético
(Antigo Testamento) e do Texto Recebido. Claro que a King James é uma
excelente tradução de ambos os Textos, servindo, portanto, como ótima
referência no estudo de determinadas passagens em que ficamos com
dúvidas no que diz respeito à tradução --- ainda mais quando não se
conhece o hebraico e o grego. E é o que faço em caráter didático para
consideração dos irmãos em vários trechos.

Acrescento esta iniciativa ao excelente e amplo ministério do
nosso querido e amado irmão Hélio de Menezes Silva, que tem prestado
um inestimável serviço em defesa do Texto Tradicional. Recomendo a todos
uma visita ao seu site: \url{http://solascriptura-tt.org}

\chapter{Pentateuco}

\section{Gênesis}
\subsection*{Gn 8.19:} 
\addcontentsline{toc}{subsection}{8.19}
\begin{quote}
    \small
Todo o animal, todo o réptil, e toda a ave, e tudo o que se move sobre a terra, conforme as suas famílias, \uwave{saiu para fora} da arca.
\end{quote}

Pleonasmo vicioso. Ver \ref{saiu}.

\subsection*{Gn 33.19} 
\addcontentsline{toc}{subsection}{33.19}:
\begin{quote}
    \small
E comprou uma parte do campo em que estendera a sua tenda, da mão dos filhos de Hamor, pai de Siquém, por cem peças de \uwave{dinheiro}.
\end{quote}

Que moeda se aplicaria aqui no original? Em outras versões, como a Edição Contemporânea, lê-se ``cem peças de prata''. Embora a King James também traduza por ``pieces of money'', sou inclinado a acreditar que no hebraico esteja ``peças de prata''. Gostaria que algum irmão pudesse conferir.

\subsection*{Gn 39.12,15:}
 \addcontentsline{toc}{subsection}{39.12,15}.
 \begin{quote}
    \small
\ldots e ela lhe pegou pela sua roupa, dizendo: Deita-te comigo. E ele deixou a sua roupa na mão dela, e fugiu, e \uwave{saiu para fora}. $^{\mathrm{15}}$E aconteceu que, ouvindo ele que eu levantava a minha voz e gritava, deixou a sua roupa comigo, e fugiu, e \uwave{saiu para fora}.
\end{quote}

Pleonasmo vicioso, novamente (\ref{saiu}).

\section{Êxodo}
\subsection*{Ex 12.10:} 
\addcontentsline{toc}{subsection}{12.10}
\begin{quote}
    \small
E nada dele deixareis \uwave{até amanhã}; mas o que dele ficar até \uwave{amanhã}, queimareis no fogo.
\end{quote}

Por favor, corrijam: é ``até a manhã''.

King James: ``And ye shall let nothing of it remain until the morning; and that which remaineth of it until the morning ye shall burn with fire.''

Edição Contemporânea: ``Nada deixareis dele até pela manhã; se algo ficar dele até pela manhã, queimareis ao fogo.''

\subsection*{Ex 14.2:} 
 \addcontentsline{toc}{subsection}{14.2}
\begin{quote}
    \small
Fala aos filhos de Israel que voltem, e que se acampem diante de Pi-Hairote, entre Migdol e o mar, diante de Baal-Zefom; em frente dele \uwave{assentareis o campo} junto ao mar.
\end{quote}

\emph{Campo} ou \emph{acampamento}? Creio que aqui cabe uma atualização para ``acampamento''. Outras versões traduzem por ``acampamento'' e a King James utiliza o verbo ``encamp'' (acampar). 

\subsection*{Ex 15.14:} 
 \addcontentsline{toc}{subsection}{15.14}
\begin{quote}
    \small
Os povos o ouviram, eles estremeceram, uma dor apoderou-se dos habitantes da \uwave{Filistia}.
\end{quote}

É \emph{Filístia} (c/ acento).

\subsection*{Ex 20.18:} 
 \addcontentsline{toc}{subsection}{20.18}
\begin{quote}
    \small
E todo o povo viu os trovões e os relâmpagos, e o sonido da buzina, e o monte fumegando; e o povo, vendo isso [,] retirou-se e pôs-se de longe.
\end{quote}

Falta uma vírgula após ``vendo isso''.

\subsection*{Ex 20.20:} 
 \addcontentsline{toc}{subsection}{20.20}
\begin{quote}
    \small
E disse Moisés ao povo: Não temais, Deus veio para vos provar, e para que o seu temor esteja diante de vós, \uwave{afim de que} não pequeis.
\end{quote}

Escreve-se ``a fim de que''.


\subsection*{Ex 37.17:} 
 \addcontentsline{toc}{subsection}{37.17}
\begin{quote}
    \small
Fez também o candelabro de ouro puro; de obra batida fez este candelabro; o seu pedestal, e as suas hastes, os seus copos, as suas \uwave{maçãs}, e as suas flores, formavam com ele uma só peça.
\end{quote}

Maçãs!? Não seriam ``botões''? Edição Contemporânea: botões. King James:
knops (botões).

\section{Levítico}
\subsection*{Lv 14.3:} 
 \addcontentsline{toc}{subsection}{14.3}
\begin{quote}
    \small
\ldots e o sacerdote \uwave{sairá fora} do arraial, e o examinará, e eis que, se a praga da lepra do leproso for sarada,\ldots
\end{quote}

Pleonasmo. Ver \ref{saiu}.

\subsection*{Lv 19.15:} 
 \addcontentsline{toc}{subsection}{19.15}
\begin{quote}
    \small
Não farás injustiça no juízo; \uwave{não respeitarás o pobre}, nem honrarás o poderoso; com justiça julgarás o teu próximo.
\end{quote}

Aqui eu me pergunto se não seria melhor a tradução por ``favorecerás'' em
vez de ``respeitarás''. Deixo a critério dos entendidos da língua. Pergunto aos irmãos qual seria o sentido a ser observado aqui?

\subsection*{Lv 23.12-13:} 
 \addcontentsline{toc}{subsection}{23.12-13}
\begin{quote}
    \small
E no dia em que moverdes o molho, preparareis um cordeiro sem defeito, de um ano, em holocausto ao Senhor, $^{\mathrm{13}}$e a sua oferta de alimentos\uline{,} será de duas dízimas de flor de farinha, amassada com azeite, para oferta queimada em cheiro suave ao Senhor, e a sua libação será de vinho, um quarto de him.
\end{quote}

Não há necessidade da vírgula após ``alimentos''.

\subsection*{Lv 24.10:} 
 \addcontentsline{toc}{subsection}{24.10}
\begin{quote}
    \small
E apareceu, no meio dos filhos de Israel[,] o filho de uma mulher israelita, o qual era filho de um homem egípcio; e o filho da israelita e um homem israelita discutiram no arraial.
\end{quote}

Falta uma vírgula após ``Israel''.

\subsection*{Lv 25.33:} 
 \addcontentsline{toc}{subsection}{25.33}
\begin{quote}
    \small
E se alguém comprar dos levitas\uwave{,} uma casa, a casa comprada e a cidade da sua possessão sairão do poder do comprador no jubileu; porque as casas das cidades dos levitas são a sua possessão no meio dos filhos de Israel.
\end{quote}

Dispensável a vírgula após ``levitas''.

\section{Números}
\subsection*{Nm 3.20:} 
 \addcontentsline{toc}{subsection}{3.20}
\begin{quote}
    \small
E os filhos de Merari pelas suas famílias: Maeli e Musi\uline{;} estas são as famílias dos levitas, segundo a casa de seus pais.
\end{quote}

Ponto em vez de ponto-e-vírgula.


\subsection*{Nm 16.45:} 
 \addcontentsline{toc}{subsection}{16.45}
\begin{quote}
    \small
Levantai-vos do meio desta congregação, e a consumirei num momento\uwave{;} então se prostraram sobre os seus rostos,\ldots
\end{quote}

Novamente, ponto e não ponto-e-vírgula.

\subsection*{Nm 18.9:} 
 \addcontentsline{toc}{subsection}{18.9}
\begin{quote}
    \small
Isto terás das coisas santíssimas do fogo\uwave{;} todas as suas ofertas com todas as suas ofertas de alimentos, e com todas as suas expiações pelo pecado, e com todas as suas expiações pela culpa, que me apresentarão; serão coisas santíssimas para ti e para teus filhos.
\end{quote}

Dois pontos em vez de ponto-e-vírgula.

\subsection*{Nm 18.16:} 
 \addcontentsline{toc}{subsection}{18.16}
\begin{quote}
    \small
Os que deles se houverem de resgatar resgatarás, da idade de um mês, segundo a tua avaliação, por cinco siclos de \uwave{dinheiro}, segundo o siclo do santuário, que é de vinte geras.
\end{quote}

Mais uma vez solicito a ajuda dos irmãos letrados em hebraico: é mais fiel traduzir-se por ``siclos de \emph{prata}'' ou por ``siclos de \emph{dinheiro}''? Exemplos: King James: ``For the money of five shekels''. Edição Contemporânea: ``siclos de prata''. Aliás, neste aspecto, pelo menos no Novo Testamento, tenho observado que a King James não é tão ``fiel'' ao grego quanto poderia na medida em que faz na
sua tradução do nome das moedas uma equivalência com sua moeda própria (inglesa). Por exemplo: Mateus 18.28: ``Saindo, porém, aquele servo, encontrou um dos seus conservos, que lhe devia cem \uwave{denários},\ldots''. Em vez de a King James traduzir literalmente por \texttt{denários}, que era o nome de uma das moedas
romanas, seus tradutores optaram por traduzir por \emph{pence}. Portanto, insisto: como está no hebraico?

\subsection*{Nm 24.3:} 
 \addcontentsline{toc}{subsection}{24.3}
\begin{quote}
    \small
E proferiu a sua parábola, e disse: Fala\uwave{,} Balaão, filho de Beor, e fala o homem de olhos abertos.
\end{quote}

Não há essa vírgula após ``Fala''. Confiram com o versículo 15: ``Então
proferiu a sua parábola, e disse: Fala Balaão, filho de Beor, e fala o
homem de olhos abertos''.

\subsection*{Nm 27.1:} 
 \addcontentsline{toc}{subsection}{27.1}
\begin{quote}
    \small
E CHEGARAM as filhas de Zelofeade, filho de Hefer, filho de Gileade, filho de Maquir, filho de Manassés, entre as famílias de Manassés, filho de José; e estes são os nomes delas\uwave{;} Maalá, Noa, Hogla, Milca, e Tirza;\ldots
\end{quote}

Dois pontos em vez de ponto-e-vírgula.

\subsection*{Nm 31.42-47:} 
 \addcontentsline{toc}{subsection}{31.42-47}
\begin{quote}
    \small
``$^{\mathrm{42}}$E da metade dos filhos de Israel que Moisés separara
da dos homens que pelejaram, $^{\mathrm{43}}$(a metade para a congregação foi, das ovelhas, trezentas e trinta e sete mil e quinhentas; $^{\mathrm{44}}$e dos bois trinta e seis mil; $^{\mathrm{45}}$e dos jumentos trinta mil e quinhentos;
$^{\mathrm{46}}$e das pessoas humanas dezesseis mil). $^{\mathrm{47}}$Desta metade dos filhos de Israel, Moisés tomou um de cada cinqüenta, de homens e de animais, e os deu aos levitas, que tinham cuidado da guarda do tabernáculo do Senhor, como o Senhor ordenara a Moisés.''
\end{quote}

Essa passagem apresenta uma redação bastante truncada. Mereceria da
parte dos redatores da Sociedade Bíblica Trinitariana do Brasil uma
boa atenção. Entre os versículos 46 e 47, por exemplo, parece ser
ponto-e-vírgula em vez de ponto. Mas não parece ser só isso.

\subsection*{Nm 34.29:} 
 \addcontentsline{toc}{subsection}{34.29}
\begin{quote}
    \small
Estes são aqueles a quem o Senhor ordenou\uline{,} que repartissem as heranças aos filhos de Israel na terra de Canaã.
\end{quote}
Não há essa vírgula após ``ordenou''.


\section{Deuteronômio}
\subsection*{Dt 14.21:} 
 \addcontentsline{toc}{subsection}{14.21}
\begin{quote}
    \small
\ldots Não cozerás o cabrito \uwave{com leite} da sua mãe.
\end{quote}

Melhor: ``no leite da sua mãe'' ou ``com o leite da sua mãe''. King James: ``Thou shalt not seethe a kid in his mother's milk''.

\subsection*{Dt 23.10:} 
 \addcontentsline{toc}{subsection}{23.10}
\begin{quote}
    \small
Quando entre ti houver alguém que, por algum acidente noturno, não estiver limpo, \uwave{sairá fora} do arraial; não entrará no meio dele.
\end{quote}

Pleonasmo vicioso. Simplesmente: ``sairá do arraial''. Ver \ref{saiu}.

\subsection*{Dt 28.67:} 
 \addcontentsline{toc}{subsection}{28.67}
\begin{quote}
    \small
Pela manhã dirás: Ah! quem me dera ver a noite! E à tarde dirás: \uwave{ah!} quem me dera ver a manhã! pelo pasmo de teu coração, que sentirás, e pelo que verás com os teus olhos.
\end{quote}

Se o primeiro ``Ah'' é maiúsculo por que não o segundo?

\subsection*{Dt 32.27:} 
 \addcontentsline{toc}{subsection}{32.27}
 \begin{quote}
    \small
 Se eu não \uwave{receiasse} a ira do inimigo, para que os seus adversários não se iludam, e para que não digam: A nossa mão está exaltada; o Senhor não fez tudo isto.
\end{quote}

Erro ortográfico: é ``receasse'', sem o `i' no meio.

\chapter{Livros Históricos}

\section{Josué}
\subsection*{Js 2.19:} 
 \addcontentsline{toc}{subsection}{2.19}
\begin{quote}
    \small
Será, pois, que qualquer que \uwave{sair fora} da porta da tua casa, o seu sangue será sobre a sua cabeça, e nós seremos inocentes; mas qualquer que estiver contigo, em casa, o seu sangue seja sobre a nossa cabeça, se alguém nele puser mão.
\end{quote}

Pleonasmo.

\subsection*{Js 7.1:} 
 \addcontentsline{toc}{subsection}{7.1}
\begin{quote}
    \small
E transgrediram os filhos de Israel no anátema; porque \uwave{Acã filho} de Carmi, filho de Zabdi, filho de Zerá, da tribo de Judá, tomou do anátema, e a ira do Senhor se acendeu contra os filhos de Israel.
\end{quote}

Falta uma vírgula depois de ``Acã''.

\subsection*{Js 7.24:} 
 \addcontentsline{toc}{subsection}{7.24}
\begin{quote}
    \small
  Então Josué, e todo o Israel com ele, tomaram a \uwave{Acã filho de Zerá}, e a prata, e a capa, e a cunha de ouro, e seus filhos, e suas filhas, e seus bois, e seus jumentos, e suas ovelhas, e sua tenda, e tudo quanto ele tinha; e levaram-nos ao vale de Acor.  
\end{quote}

Novamente, depois de ``Acã'' falta uma vírgula.

\subsection*{Js 19.46-47:} 
 \addcontentsline{toc}{subsection}{19.46-47}
\begin{quote}
    \small
\ldots e Me-Jarcom, e Racom, com o termo defronte de Jafo\uline{;} $^{\mathrm{47}}$saiu, porém, pequeno termo aos filhos de Dã, pelo que subiram os filhos de Dã, e pelejaram contra Lesém, e a tomaram, e a feriram ao fio da espada, e a possuíram e habitaram nela; e a Lesém chamaram Dã, conforme ao nome de Dã seu pai.    
\end{quote}

Ponto em vez de ponto-e-vírgula entre um versículo e outro.

\section{Juízes}
\subsection*{Jz 1.21-22:} 
 \addcontentsline{toc}{subsection}{1.21-22}
\begin{quote}
    \small
Porém os filhos de Benjamim não expulsaram os jebuseus que habitavam em Jerusalém; antes os jebuseus ficaram habitando com os filhos de Benjamim em Jerusalém, até ao dia de hoje\uline{,} $^{\mathrm{22}}$e subiu também a casa de José contra Betel, e foi o Senhor com eles.
\end{quote}

Não vírgula, mas ponto final, entre os versículos 21 e 22.

\subsection*{Jz 6.30:} 
 \addcontentsline{toc}{subsection}{6.30}
\begin{quote}
    \small
    Então os homens daquela cidade disseram a Joás: Tira para fora a teu filho\uline{;} para que morra\uline{;} pois derribou o altar de Baal\uline{,} e cortou o bosque que estava ao pé dele.
\end{quote}

Vírgulas em vez de ponto-e-vírgulas. Sem vírgula após ``Baal''.

\subsection*{Jz 9.7-8:} 
 \addcontentsline{toc}{subsection}{9.7-8}
\begin{quote}
    \small
    E, dizendo-o a Jotão, foi e pôs-se no cume do monte de Gerizim, e levantou a sua voz, e clamou e disse-lhes: Ouvi-me, cidadãos de Siquém, e Deus vos ouvirá a vós\uline{;} $^{\mathrm{8}}$foram uma vez as árvores a ungir para si um rei, e disseram à oliveira: Reina tu sobre nós.
\end{quote}

Ponto em vez de ponto-e-vírgula entre os versículos 7 e 8.

\subsection*{Jz 20.46:} 
 \addcontentsline{toc}{subsection}{20.46}
\begin{quote}
    \small
   E\uline{,} todos os que caíram de Benjamim, naquele dia, foram vinte e cinco mil homens que tiravam a espada, todos eles homens valentes. 
\end{quote}

Tirar a primeira vírgula.

\subsection*{Jz 21.13:} \label{enviar}
 \addcontentsline{toc}{subsection}{21.13}
\begin{quote}
    \small
    Então toda a assembléia \uwave{enviou}, e falou aos filhos de Benjamim, que estavam na penha de Rimom, e lhes proclamou a paz.
\end{quote}

Enviou o quê? Para quem?

Edição Contemporânea: ``Então toda a congregação enviou mensageiros aos filhos de Benjamin, \ldots''.

King James: ``And the whole congregation sent \emph{some} to speak to the
children of Benjamin that \emph{were} in the rock Rimmon, and to call
peaceably unto them.''

\section{I Samuel}
\subsection*{I Sm 9.26:} 
 \addcontentsline{toc}{subsection}{9.26}
\begin{quote}
    \small
E se levantaram de madrugada; e sucedeu que, quase ao subir da alva, chamou Samuel a Saul ao eirado, dizendo: Levanta-te, e despedir-te-ei. Levantou-se Saul, e \uwave{saíram ambos para fora}, ele e Samuel.
\end{quote}

Pleonasmo (\ref{saiu}).

\subsection*{I Sm 15.12:} 
 \addcontentsline{toc}{subsection}{15.12}
\begin{quote}
    \small
E madrugou Samuel para encontrar a Saul pela manhã\uline{:} e anunciou-se a Samuel, dizendo: Já chegou Saul ao Carmelo, e eis que levantou para si uma coluna. Então voltando, passou e desceu a Gilgal.
\end{quote}

Vírgula, não dois pontos, após ``manhã''.

\subsection*{I Sm 19.17:} 
 \addcontentsline{toc}{subsection}{19.17}
\begin{quote}
    \small
Então disse Saul a Mical: Por que assim me enganaste, e deixaste ir e escapar o meu inimigo? E disse Mical a Saul: Porque ele me disse: Deixa-me ir, \uwave{por que hei de eu matar-te}?
\end{quote}

Melhor: ``Deixa-me ir, por que (eu) hei de te matar?''

\subsection*{I Sm 22.16-17:} 
 \addcontentsline{toc}{subsection}{22.16-17}
\begin{quote}
    \small
$^{\mathrm{16}}$Porém o rei disse: Aimeleque, morrerás certamente, tu e toda a casa de teu pai\uline{,} $^{\mathrm{17}}$e disse o rei aos da sua guarda que estavam com ele: Virai-vos, e matai os sacerdotes do Senhor, porque também a sua mão é com Davi, e porque souberam que fugiu e não mo fizeram saber. Porém os criados do rei não quiseram estender as suas mãos para arremeter contra os sacerdotes do Senhor.
\end{quote}

É ponto entre um versículo e outro, não vírgula.

\section{II Samuel}
\subsection*{II Sm 19.14:} 
 \addcontentsline{toc}{subsection}{19.14}
\begin{quote}
    \small
Assim moveu ele o coração de todos os homens de Judá, como o de um só homem; e \uwave{enviaram} ao rei, dizendo: Volta tu com todos os teus servos.
\end{quote}

Enviaram o quê? (ver também \ref{enviar}).

King James: ``And he bowed the heart of all the men of Judah, even as the heart of one man; so that they \uline{sent this word} unto the king, return thou, and all thy servants.''

\subsection*{II Sm 21.8:} 
 \addcontentsline{toc}{subsection}{21.8}
\begin{quote}
    \small
Mas tomou o rei os dois filhos de Rispa, filha \uwave{da} Aiá, que tinha tido de Saul, a Armoni e a Mefibosete; como também os cinco filhos da irmã de Mical, filha de Saul, que tivera de Adriel, filho de Barzilai, meolatita,\ldots
\end{quote}

``DE Aiá'' (cf. v.10: Então Rispa, filha de Aiá, tomou um pano de cilício\ldots).


\subsection*{II Sm 23.33:} 
 \addcontentsline{toc}{subsection}{23.33}
\begin{quote}
    \small
Samá, \uwave{hararita}, Aião, filho de Sarar, \uwave{ararita};\ldots
\end{quote}

Hararita ou ararita? Na Revista e Atualizada e na Revista e Corrigida,
está igual. King James: ``Shammah the Hararite, Ahiam the son of
Sharar the Hararite''. Na Bíblia de Jerusalém: ``Jônatas, filho de
Sama, de Arar. Aiam, filho de Sarar, de Arar.'' Parece-me ser Hararita
nos dois casos.

\section{I Reis}
\subsection*{I Rs 4.15-19:} 
 \addcontentsline{toc}{subsection}{4.15-19}
\begin{quote}
    \small
$^{\mathrm{15}}$Aimaás em Naftali; também este tomou a Basemate, filha de Salomão, por mulher; $^{\mathrm{16}}$Baaná, filho de Husai, em Aser e em Alote; $^{\mathrm{17}}$Jeosafá, filho de Parua, em Issacar; $^{\mathrm{18}}$Simei, filho de Elá, em Benjamim\uline{:} $^{\mathrm{19}}$Geber, filho de Uri, na terra de Gileade, a terra de Siom, rei dos amorreus, e de Ogue, rei de Basã; e só uma guarnição havia naquela terra.
\end{quote}

Ponto-e-vírgula em lugar de dois pontos, entre os versículos 18 e 19.

\subsection*{I Rs 6.4:} 
 \addcontentsline{toc}{subsection}{6.4}
\begin{quote}
    \small
E fez para a casa janelas de \uwave{gelósias} fixas.
\end{quote}

Segundo o dicionário: \textbf{gelosias}, sem acento.

\subsection*{I Rs 7.51:} 
 \addcontentsline{toc}{subsection}{7.51}
\begin{quote}
    \small
Assim se acabou toda a obra que fez o rei Salomão para a casa do Senhor\uline{;} então trouxe Salomão as coisas que seu pai Davi havia consagrado; a prata, e o ouro, e os objetos pôs entre os tesouros da casa do Senhor.
\end{quote}

Ponto em lugar de ponto-e-vírgula.

\subsection*{IRs 8.18:} 
 \addcontentsline{toc}{subsection}{8.18}
\begin{quote}
    \small
Porém o Senhor disse a Davi, meu pai: Porquanto propuseste no teu coração o edificar casa ao meu nome[,] bem fizeste em o propor no teu coração.
\end{quote}

Falta a vírgula indicada.

\subsection*{IRs 8.35:} 
 \addcontentsline{toc}{subsection}{8.35}
\begin{quote}
    \small
\uwave{Quando os céus se fechar}, e não houver chuva, por terem pecado contra ti, e orarem neste lugar, e confessarem o teu nome, e se converterem dos seus pecados, havendo-os tu afligido,\ldots
\end{quote}

``Quando os céus se FECHAREM''. Aqui um erro crasso de concordância verbal.

\subsection*{ IRs 9.5-6:} 
 \addcontentsline{toc}{subsection}{9.5-6}
\begin{quote}
    \small
$^{\mathrm{5}}$\ldots então confirmarei o trono de teu reino sobre Israel para sempre; como falei acerca de teu pai Davi, dizendo: Não te faltará sucessor sobre o trono de Israel\uline{;} $^{\mathrm{6}}$porém, se vós e vossos filhos de qualquer maneira vos apartardes de mim, e não guardardes os meus mandamentos, e os meus estatutos, que vos tenho proposto, mas fordes, e servirdes a outros deuses, e vos prostrardes perante eles,\ldots
\end{quote}

Ponto entre um versículo e outro, não ponto-e-vírgula.

\subsection*{I Rs 12.2-3:} 
 \addcontentsline{toc}{subsection}{12.2-3}
\begin{quote}
    \small
$^{\mathrm{2}}$Sucedeu que, Jeroboão, filho de Nebate, achando-se ainda no Egito, para onde fugira de diante do rei Salomão, voltou do Egito, $^{\mathrm{3}}$porque mandaram chamá-lo\uwave{;} veio, pois, Jeroboão e toda a congregação de Israel, e falaram a Roboão, dizendo:\ldots
\end{quote}

Ponto, não ponto-e-vírgula, após ``chamá-lo''.

\subsection*{I Rs 19.11a,13:} 
 \addcontentsline{toc}{subsection}{19.11a,13}
\begin{quote}
    \small
E Deus lhe disse: \uwave{Sai para fora}, e põe-te neste monte perante o Senhor. $^{\mathrm{13}}$E sucedeu que, ouvindo-a Elias, envolveu o seu rosto na sua capa, e \uwave{saiu para fora}, e pôs-se à entrada da caverna; e eis que veio a ele uma voz, que dizia: Que fazes aqui, Elias?
\end{quote}

Dois casos seguidos de pleonasmo vicioso (\ref{saiu}).

\subsection*{I Rs 21.2:} 
 \addcontentsline{toc}{subsection}{21.2}
\begin{quote}
    \small
Então Acabe falou a Nabote, dizendo: Dá-me a tua vinha, para que me sirva de horta, pois está vizinha ao lado da minha casa; e te darei por ela outra vinha melhor\uline{:} ou, se for do teu agrado, dar-te-ei o seu valor em dinheiro.
\end{quote}

Ponto-e-vírgula em vez de dois pontos.

\section{II Reis}
\subsection*{II Rs 7.11:} 
 \addcontentsline{toc}{subsection}{7.11}
\begin{quote}
    \small
E chamaram os porteiros, e o anunciaram dentro da casa do rei.
\end{quote}

Nessa redação, do jeito que está, parece que quem ``chamaram os porteiros'' foram os mesmos que fizeram o anúncio.

King James: ``And he called the porters; and they told it to the king's house within''.

Melhor ficaria: ``E chamaram os porteiros, e \emph{eles} o anunciaram dentro da casa do rei''.

\subsection*{II Rs 11.19-21:} 
 \addcontentsline{toc}{subsection}{11.19-21}
\begin{quote}
    \small
E tomou os centuriões, e os capitães, e os da guarda, e todo o povo da terra; e conduziram da casa do Senhor\uline{,} o rei, e foram, pelo caminho da porta dos da guarda, à casa do rei, e ele se assentou no trono dos reis. $^{\mathrm{20}}$E todo o povo da terra se alegrou, e a cidade repousou, depois que mataram a Atalia, à espada, junto à casa do rei\uline{,} $^{\mathrm{21}}$era Joás da idade de sete anos quando o fizeram rei.
\end{quote}

Tirar a primeira vírgula em destaque (mal-colocada após ``Senhor''), que faz do rei em questão o próprio Senhor. Ou então que se acrescente outra antes, ficando: ``e conduziram, da casa do Senhor, o rei''. Ainda: entre os versículos 20 e 21 é ponto, não vírgula.

\subsection*{II Rs 17.11:} 
 \addcontentsline{toc}{subsection}{17.11}
\begin{quote}
    \small
E queimaram ali incenso em todos os altos, como as nações\uline{,} que o Senhor expulsara de diante deles; e fizeram coisas ruins, para provocarem à ira o Senhor.
\end{quote}

Não há essa vírgula após ``nações''.

\subsection*{II Rs 17.25:} 
 \addcontentsline{toc}{subsection}{17.25}
\begin{quote}
    \small
E sucedeu que, no princípio da sua habitação ali, não temeram ao Senhor; e o Senhor mandou entre eles\uline{,} leões, que mataram a alguns deles.
\end{quote}

Não há vírgula depois de ``eles''. Ou que se intercale com vírgulas
``entre eles''.

\subsection*{II Rs 19.29:} 
 \addcontentsline{toc}{subsection}{19.29}
\begin{quote}
    \small
E isto te será por sinal\uline{;} este ano se comerá o que nascer por si mesmo, e no ano seguinte o que daí proceder; porém, no terceiro ano semeai e segai, plantai vinhas\uline{,} e comei os seus frutos.
\end{quote}

Dois pontos em vez de ponto-e-vírgula. Ainda: não há vírgula após ``plantai vinhas''. Ou fica: ``\ldots plantai vinhas e comei os seus frutos'' ou ``\ldots \textbf{e} plantai vinhas, \textbf{e} comei os seus frutos''.

\subsection*{II Rs 22.3:} 
 \addcontentsline{toc}{subsection}{22.3}
\begin{quote}
    \small
Sucedeu que, no ano décimo oitavo do rei Josias, o rei \uwave{mandou ao} escrivão Safã, filho de Azalias, filho de Mesulão, à casa do Senhor, dizendo:\ldots
\end{quote}

``Mandou o escrivão'', não ``ao escrivão''.

RA: No décimo oitavo ano do seu reinado, o rei Josias mandou o escrivão Safã, filho de Azalias, filho de Mesulão, à Casa do SENHOR,\ldots

RC: Sucedeu, pois, que, no ano décimo oitavo do rei Josias, o rei mandou o escrivão Safã, filho de Azalias, filho de Mesulão, à Casa do SENHOR, dizendo: \ldots

\section{I Crônicas}
\subsection*{I Cr 3:12:} 
 \addcontentsline{toc}{subsection}{3.12}
\begin{quote}
    \small
De quem foi filho Amazias; de quem foi filho Jotão;\ldots
\end{quote}

```De quem foi filho Amazias; DE QUEM FOI FILHO AZARIAS, de quem foi filho Jotão;\ldots'

Como no T. Massorético, na KJV (e até mesmo na ARAtualizada!). Lembrar que Azarias = Ozias = Uzias.'' (Hélio de Menezes Silva).

\subsection*{I Cr 3.15:} 
 \addcontentsline{toc}{subsection}{3.15}
\begin{quote}
    \small
E os filhos de Josias foram: o primogênito, Joanã\uline{:} o segundo, Jeoiaquim; o terceiro, Zedequias; o quarto, Salum.
\end{quote}

Em vez de dois pontos, ponto-e-vírgula.

\subsection*{I Cr 3.24:} 
 \addcontentsline{toc}{subsection}{3.24}
\begin{quote}
    \small
E os filhos de Elioenai\uline{;} Hodavias, Eliasibe, Pelaías, Acube, Joanã, Delaías\uline{,} e Anani, sete.
\end{quote}

Aqui, o contrário: em vez de ponto-e-vírgula, dois pontos. Também não
há vírgula após ``Delaías''.

\subsection*{I Cr 4.42:} 
 \addcontentsline{toc}{subsection}{4.42}
\begin{quote}
    \small
Também deles, dos filhos de Simeão, quinhentos homens foram às montanhas de \uwave{Seír}; levaram por cabeças a Pelatias, e a Nearias, e a Refaías, e a Uziel, filhos de Isi.
\end{quote}

``Seir'', sem acento.

\subsection*{I Cr 8.6-7:} 
 \addcontentsline{toc}{subsection}{8.6-7}
\begin{quote}
    \small
E estes foram os filhos de Eúde; que foram chefes dos pais dos moradores de Geba, e os levaram cativos a Manaate\uline{;} $^{\mathrm{7}}$e Naamã, e Aías e Gera; este os transportou, e gerou a Uzá e a Aiúde.
\end{quote}

Em vez de ponto-e-vírgula, entre os versículos, dois pontos.

\subsection*{I Cr 9.13:} 
 \addcontentsline{toc}{subsection}{9.13}
\begin{quote}
    \small
\ldots como também seus irmãos, cabeças nas casas de seus pais, mil\uline{,} setecentos e sessenta, homens valentes para a obra do ministério da casa de Deus.
\end{quote}

Não há vírgula após o milhar: mil setecentos e sessenta.

\subsection*{I Cr 12.18:} 
 \addcontentsline{toc}{subsection}{12.18} Então veio o \uwave{espírito} sobre Amasai, chefe de trinta, e disse: Nós somos teus, ó Davi, e contigo estamos, ó filho de Jessé! Paz, paz contigo, e paz com quem te ajuda, pois que teu Deus te ajuda. E Davi os recebeu, e os fez capitães das tropas.

Aqui uma dúvida: a King James também registra ``espírito''. Outras
versões: ``Espírito''. Se não for Espírito (Santo, de Deus), que
espírito veio sobre Amasai? Será que no ``original'' a idéia é de
inspiração ou algo semelhante?

\subsection*{I Cr 15.27-28:} 
 \addcontentsline{toc}{subsection}{15.27-28}
\begin{quote}
    \small
E Davi ia vestido de um manto de linho fino, como também todos os levitas que levavam a arca, e os cantores, e Quenanias, mestre dos cantores; também Davi levava sobre si um éfode de linho\uline{,} $^{\mathrm{28}}$e todo o Israel fez subir a arca da aliança do Senhor, com júbilo, e ao som de buzinas, e de trombetas, e de címbalos, fazendo ressoar alaúdes e harpas.
\end{quote}

Ponto, em vez de vírgula, entre os versículos 27 e 28.

\subsection*{I Cr 16.7-8:} 
 \addcontentsline{toc}{subsection}{16.7-8}
\begin{quote}
    \small
Então naquele mesmo dia Davi, em primeiro lugar, deu o seguinte salmo para que, pelo ministério de Asafe e de seus irmãos, louvassem ao Senhor\uline{;} $^{\mathrm{8}}$\uline{l}ouvai ao Senhor, invocai o seu nome, fazei conhecidas as suas obras entre os povos.
\end{quote}

Em vez de ponto-e-vírgula, dois pontos: ``\ldots ao Senhor: $^{\mathrm{8}}$Louvai ao Senhor, \ldots''.


\subsection*{I Cr 19.1:} 
 \addcontentsline{toc}{subsection}{19.1}
\begin{quote}
    \small
 E ACONTECEU, \uwave{depois disto}[,] que Naás, rei dos filhos de Amom, morreu; e seu filho reinou em seu lugar.
 \end{quote}

Acrescentar a vírgula após ``depois disto’’. Ainda: se se refere ao que já foi dito, usa-se ``isso’’; logo: ``depois disso’’, não ``disto’’.

\subsection*{I Cr 19.3:} 
 \addcontentsline{toc}{subsection}{19.3}
\begin{quote}
    \small
\ldots disseram os príncipes dos filhos de Amom a Hanum: Pensas[,] porventura, que foi para honrar teu pai aos teus olhos, que Davi te mandou consoladores? Não vieram seus servos a ti, a esquadrinhar, e a transtornar, e a espiar a terra?
\end{quote}

Acrescentar a vírgula após ``Pensas’’.

\subsection*{I Cr 20.7-8:} 
 \addcontentsline{toc}{subsection}{20.7-8}
\begin{quote}
    \small
E injuriou a Israel; porém Jônatas, filho de Simei, irmão de Davi, o feriu\uline{;} $^{\mathrm{8}}$estes nasceram ao gigante em Gate; e caíram pela mão de Davi e pela mão dos seus servos.
\end{quote}
 
Ponto entre um versículo e outro, não ponto-e-vírgula.

\subsection*{I Cr 22.13:} 
 \addcontentsline{toc}{subsection}{22.13}
\begin{quote}
    \small
 Então prosperarás, se tiveres cuidado de cumprir os estatutos e os juízos, que o Senhor mandou a Moisés acerca de Israel\uline{;} esforça-te, e tem bom ânimo; não temas, nem tenhas pavor.
\end{quote}
 
Ponto em lugar de ponto-e-vírgula. King James: dois pontos.

\subsection*{I Cr 23.11:} 
 \addcontentsline{toc}{subsection}{23.11}
\begin{quote}
    \small
E Jaate era o chefe, e Ziza o segundo, mas Jeús e Berias não tiveram muitos filhos; por isso estes, sendo contados juntos [,] se tornaram uma só família.
Acrescentar a vírgula após ``juntos’’.
\end{quote}

\subsection*{I Cr 28.2:} 
 \addcontentsline{toc}{subsection}{28.2}

E pôs-se o rei Davi em pé, e disse: Ouvi-me, irmãos meus, e povo meu\uline{;} em meu coração \uwave{propus eu} edificar uma casa de repouso para a arca da aliança do Senhor e para o estrado dos pés do nosso Deus, e eu tinha feito o preparo para a edificar.
 
Primeiramente, ``propus edificar’’, sem o pronome ``eu’’, cabe
melhor. E, em segundo lugar, dois pontos em vez de ponto-e-vírgula: ``Ouvi-me,
irmãos meus, e povo meu\textbf{:} em meu coração\ldots’’.

\section{II Crônicas}
\subsection*{II Cr 2.7:} 
 \addcontentsline{toc}{subsection}{2.7}
\begin{quote}
    \small
Manda-me, pois, agora [,] um homem hábil para trabalhar em ouro, em prata, em bronze, em ferro, em púrpura, em carmesim e em azul; e que saiba lavrar ao buril, juntamente com os peritos que estão comigo em Judá e em Jerusalém, os quais Davi, meu pai, preparou.
 \end{quote}
 
Falta a vírgula no lugar acima indicado.

\subsection*{II Cr 3.14-15:} 
 \addcontentsline{toc}{subsection}{3.14-15}
\begin{quote}
    \small
Também fez o véu de azul, púrpura, carmesim e linho fino; e pôs sobre ele querubins\uline{;} $^{\mathrm{15}}$fez também, diante da casa, duas colunas de trinta e cinco côvados de altura; e o capitel, que estava sobre cada uma, era de cinco côvados.
 \end{quote}
 
Não ponto-e-vírgula, mas ponto entre um versículo e outro.

\subsection*{II Cr 5.11-14:} 
 \addcontentsline{toc}{subsection}{5.11-14}
\begin{quote}
    \small
E sucedeu que, saindo os sacerdotes do santuário (porque todos os sacerdotes, que ali se acharam, se santificaram, sem respeitarem as suas turmas, $^{\mathrm{12}}$e os levitas, que eram cantores, todos eles, de Asafe, de Hemã, de Jedutum, de seus filhos e de seus irmãos, vestidos de linho fino, com címbalos, com saltérios e com harpas, estavam em pé para o oriente do altar; e com eles até cento e vinte sacerdotes, que tocavam as trombetas). $^{\mathrm{13}}$E aconteceu que, quando eles uniformemente tocavam as trombetas, e cantavam, para fazerem ouvir uma só voz, bendizendo e louvando ao Senhor; e levantando eles a voz com trombetas, címbalos, e outros instrumentos musicais, e louvando ao Senhor, dizendo: Porque ele é bom, porque a sua benignidade dura para sempre, então a casa se encheu de uma nuvem, a saber, a casa do Senhor; $^{\mathrm{14}}$e os sacerdotes não podiam permanecer em pé, para ministrar, por causa da nuvem; porque a glória do Senhor encheu a casa de Deus.
\end{quote}

Redação muito ruim, truncada demais. Entre os versículos 12 e 13
não pode ser ponto. O texto entre parênteses está longo demais. Há que
se dar mais ``fluidez’’ nessa passagem. Só posso pedir aos editores
da Sociedade Bíblica Trinitariana do Brasil que revisem esse trecho e
dêem melhor coerência textual.

Vejam como o mesmo trecho editado pela Sociedade Bíblica
do Brasil, em sua versão Revista e Corrigida, está bem melhor redigido:

RC: E sucedeu que, saindo os sacerdotes do santuário (porque todos os
sacerdotes, que se acharam, se santificaram, sem guardarem as suas
turmas), $^{\mathrm{12}}$e quando os levitas, cantores de todos eles,
isto é, Asafe, Hemã, Jedutum, seus filhos e seus irmãos, vestidos de
linho fino, com címbalos, e com alaúdes e com harpas, estavam em pé
para o oriente do altar, e com eles até cento e vinte sacerdotes, que
tocavam as trombetas, $^{\mathrm{13}}$e quando eles uniformemente
tocavam as trombetas e cantavam para fazerem ouvir uma só voz,
bendizendo e louvando ao SENHOR, e quando levantavam eles a voz com
trombetas, e címbalos, e outros instrumentos músicos, para bendizerem
ao SENHOR, porque era bom, porque a sua benignidade durava para
sempre, então, a casa se encheu de uma nuvem, a saber, a Casa do
SENHOR; $^{\mathrm{14}}$e não podiam os sacerdotes ter-se em pé, para
ministrar, por causa da nuvem, porque a glória do SENHOR encheu a Casa
de Deus.

\subsection*{II Cr 6.7:} 
 \addcontentsline{toc}{subsection}{6.7}
\begin{quote}
    \small
Também Davi [,] meu pai [,] teve no seu coração o edificar uma casa ao nome do Senhor Deus de Israel.
 \end{quote}
 
Faltam as vírgulas indicadas.

\subsection*{II Cr 6.36:} 
\addcontentsline{toc}{subsection}{6.36}
\begin{quote}
    \small
Quando pecarem contra ti (pois não há homem que não peque), e tu te indignares contra eles, e os entregares diante do inimigo, \uwave{para que os que os cativarem os levem em cativeiro} para alguma terra, remota ou vizinha,\ldots
 \end{quote}
 
Irmãos, achei digno de nota a frase em destaque, mesmo esquisita ou um
tanto confusa: ``para que os que os cativarem os levem em
cativeiro’’?! Não seria, simplesmente: ``para que os levem cativos a
uma terra, longe ou perto’’ (Edição Contemporânea)?

King James: ``If they sin against thee, (for \emph{there is} no man which sinneth not,) and thou be angry with them, and deliver them over before \emph{their} enemies, and they carry them away captives unto a land far off or near; \ldots’’.

\subsection*{II Cr 14.15:} 
 \addcontentsline{toc}{subsection}{14.15}
\begin{quote}
    \small
 Também feriram as \uwave{malhadas do gado}; e levaram ovelhas em abundância, e camelos, e voltaram para Jerusalém.
 \end{quote}

Gostaria de entender melhor essa tradução por ``malhadas do gado’’ da ACFiel. A King James quase concorda com a Revista e Atualizada quando registra ``tendas do gado’’. Pergunto aos irmãos: essa tradução por ``malhadas do gado’’ é, realmente, a mais fiel ao Texto Massorético?

RA: Também feriram as tendas dos donos do gado, levaram ovelhas em abundância e camelos e voltaram para Jerusalém.

Edição Contemporânea: ``acampamento dos pastores’’.

KJ: They smote also the tents of cattle, and carried away sheep and camels in abundance, and returned to Jerusalem.

\subsection*{II Cr 18.18:} 
 \addcontentsline{toc}{subsection}{18.18}
\begin{quote}
    \small
Disse mais: Ouvi, pois, a palavra do Senhor:  \uwave{Vi ao Senhor} assentado no seu trono, e todo o exército  celestial em pé à sua mão direita, e à sua esquerda.
\end{quote}
 
``Vi \emph{ao} Senhor’’: português antigo, influência  espanhola. Melhor: ``Vi \textbf{o} Senhor’’.

\subsection*{II Cr 24.11-12:} 
 \addcontentsline{toc}{subsection}{24.11-12}
\begin{quote}
    \small
E sucedia que, quando levavam o cofre pelas mãos dos levitas, segundo o mandado do rei, e vendo-se que já havia muito dinheiro, vinha o escrivão do rei, e o oficial do sumo sacerdote, e esvaziavam o cofre, e tomavam-no e levavam-no de novo ao seu lugar; assim faziam de dia em dia, e ajuntaram dinheiro em abundância\uline{.} $^{\mathrm{12}}$O qual o rei e Joiada davam aos que tinham o encargo da obra do serviço da casa do Senhor; e contrataram pedreiros e carpinteiros, para renovarem a casa do Senhor; como também ferreiros e serralheiros, para repararem a casa do Senhor.
 \end{quote}
 
Vírgula, simplesmente, não ponto, entre um versículo e outro --- pelo menos diante de tal tradução da passagem.

RC: $^{\mathrm{11}}$E sucedeu que, ao tempo que traziam a arca pelas mãos dos levitas, segundo o mandado do rei, e vendo que já havia muitodinheiro, vinham o escrivão do rei e o comissário do sumo sacerdote, e esvaziavam a arca, e a tomavam, e a tornavam ao seu lugar; assim faziam dia após dia e ajuntaram dinheiro em abundância, $^{\mathrm{12}}$o qual o rei e Joiada davam aos que dirigiam a obra do
serviço da Casa do SENHOR e alugaram pedreiros e carpinteiros, para renovarem a Casa do SENHOR, como também ferreiros e serralheiros, para repararem a Casa do SENHOR.

Na King James, por exemplo, bem como na Edição Contemporânea, é ponto
entre os versículos 11 e 12, mas o estilo usado na tradução contribui
para isso --- o que não é o caso em questão.


\subsection*{II Cr 24.26:} 
 \addcontentsline{toc}{subsection}{24.26}
\begin{quote}
    \small
Estes, pois, foram os que conspiraram contra ele\uwave{;} Zabade, filho de Simeate, a amonita, e Jeozabade, filho de Sinrite, a moabita.
\end{quote}
 
Dois pontos em vez de ponto-e-vírgula.

\textbf{RA}: Sepultaram-no na Cidade de Davi, porém não nos sepulcros dos reis. Foram estes os que conspiraram contra ele: Zabade, filho de Simeate, a amonita, e Jeozabade, filho de Sinrite, a moabita.

\textbf{RC}: Estes, pois, foram os que conspiraram contra ele: Zabade, filho de Simeate, a amonita, e Jozabade, filho de Sinrite, a moabita.

Porém, a \textbf{KJ} também usa o ponto-e-vírgula: And these are they that conspired against him; Zabad the son of Shimeath an Ammonitess, and Jehozabad the son of Shimrith a Moabitess. 

\subsection*{II Cr 25.23:} 
 \addcontentsline{toc}{subsection}{25.23}
\begin{quote}
    \small
E Jeoás, rei de Israel, prendeu a Amazias, rei de Judá, filho de Joás, \uwave{o} filho de Jeoacaz, em Bete-Semes, e o trouxe a Jerusalém; e derrubou o muro de Jerusalém, desde a porta de Efraim até à porta da esquina, quatrocentos côvados.
\end{quote}
 
Não está demais este ``o’’? Ou se põe o artigo ante os dois filhos ou
em nenhum deles (cf. King James).

\textbf{RA}: E Jeoás, rei de Israel, prendeu a Amazias, rei de Judá, filho de Joás, filho de Jeoacaz, em Bete-Semes; levou-o a Jerusalém, cujo muro ele rompeu desde a Porta de Efraim até à Porta da Esquina, quatrocentos côvados.

\textbf{RC}: E Joás, rei de Israel, prendeu a Amazias, rei de Judá, filho de Joás, o filho de Jeoacaz, em Bete-Semes, e o trouxe a Jerusalém; e derribou o muro de Jerusalém, desde a Porta de Efraim até à Porta da Esquina, um trecho de quatrocentos côvados.

\textbf{KJ}: And Joash the king of Israel took Amaziah king of Judah, the son of Joash, the son of Jehoahaz, at Bethshemesh, and brought him to Jerusalem, and brake down the wall of Jerusalem from the gate of Ephraim to the corner gate, four hundred cubits.

\subsection*{II Cr 32.21:} 
 \addcontentsline{toc}{subsection}{32.21}
\begin{quote}
    \small
Então o Senhor enviou um anjo que destruiu a todos os homens valentes, e os líderes, e os capitães no arraial \uwave{do rei da Assíria; e envergonhado} voltou à sua terra; e, entrando na casa de seu deus, alguns dos seus próprios filhos\uline{,} o mataram ali à espada.
 \end{quote}

Primeiramente, esclarecer uma ambiguidade: quem ficou envergonhado? O
anjo do Senhor ou o rei da Assíria? Claro que o rei da Assíria.
Portanto: ``\emph{e este}, envergonhado\ldots’’. Por fim, não há a
última vírgula.

\textbf{RA}: Então, o SENHOR enviou um anjo que destruiu todos os homens valentes, os chefes e os príncipes no arraial do rei da Assíria; e este, com o rosto coberto de vergonha, voltou para a sua terra. Tendo ele entrado na casa de seu deus, os seus próprios filhos ali o mataram à espada.

\textbf{RC}: Então, o SENHOR enviou um anjo que destruiu todos os varões valentes, e os príncipes, e os chefes no arraial do rei da Assíria; e este tornou com vergonha de rosto à sua terra; e, entrando na casa de seu deus, os mesmos que descenderam dele o mataram ali à espada.

King James: ``And the LORD sent an angel, which cut off all the mighty men of valour, and the leaders and captains in the camp of the king of Assyria. So he returned with shame of face to his own land. And when he was come into the house of his god, they that came forth of his own bowels slew him there with the sword.’’

\subsection*{II Cr 36.15:} 
 \addcontentsline{toc}{subsection}{36.15}
\begin{quote}
    \small
E o Senhor Deus de seus pais\uline{,} falou-lhes constantemente por intermédio dos mensageiros, porque se compadeceu do seu povo e da sua habitação.
\end{quote}
 
Ou se tira a vírgula indicada ou se acrescenta outra após ``Senhor’’: ``E o Senhor, Deus de seus pais, falou-lhes \ldots’’.

\section{Esdras}
\subsection*{Ed 9.13-14:} 
 \addcontentsline{toc}{subsection}{9.13-14}
\begin{quote}
    \small
E depois de tudo o que nos tem sucedido por causa das nossas más obras, e da nossa grande culpa, porquanto tu, ó nosso Deus, impediste que fôssemos destruídos, por causa da nossa iniqüidade, e ainda nos deste um remanescente como este\uwave{.} $^{\mathrm{14}}$Tornaremos, pois, agora a violar os teus mandamentos e a aparentar-nos com os povos destas abominações? Não te indignarias tu assim contra nós até de todo nos consumir, até que não ficasse remanescente nem quem escapasse?
\end{quote}

Não é ponto entre um versículo e outro, mas vírgula.

\textbf{RA}: $^{\mathrm{13}}$Depois de tudo o que nos tem sucedido por causa
das nossas más obras e da nossa grande culpa, e vendo ainda que tu, ó
nosso Deus, nos tens castigado menos do que merecem as nossas
iniqüidades e ainda nos deste este restante que escapou,
$^{\mathrm{14}}$tornaremos a violar os teus mandamentos e a
aparentar-nos com os povos destas abominações? Não te indignarias tu,
assim, contra nós, até de todo nos consumires, até não haver restante
nem alguém que escapasse?

\textbf{RC}: $^{\mathrm{13}}$E, depois de tudo o que nos tem sucedido por causa
das nossas más obras e da nossa grande culpa, ainda assim tu, ó nosso
Deus, estorvaste que fôssemos destruídos, por causa da nossa
iniqüidade, e ainda nos deste livramento como este,
$^{\mathrm{14}}$tornaremos, pois, agora a violar os teus mandamentos e
a aparentar-nos com os povos destas abominações? Não te indignarias
tu, assim, contra nós até de todo nos consumires, até que não ficasse
resto nem quem escapasse?

\textbf{KJ}: $^{\mathrm{13}}$And after all that is come upon us for our evil
deeds, and for our great trespass, seeing that thou our God hast
punished us less than our iniquities deserve, and hast given us such
deliverance as this; $^{\mathrm{14}}$should we again break thy
commandments, and join in affinity with the people of these
abominations? wouldest not thou be angry with us till thou hadst
consumed us, so that there should be no remnant nor escaping?

\section{Neemias}
\subsection*{Ne 2.18:} 
 \addcontentsline{toc}{subsection}{2.18}
\begin{quote}
    \small
Então lhes declarei como a mão do meu Deus me fora favorável, como também as palavras do rei, que ele me tinha dito\uline{;} então disseram: Levantemo-nos, e edifiquemos. E esforçaram as suas mãos para o bem.
\end{quote}

Ponto em lugar de ponto-e-vírgula.

\textbf{RC}: Então, lhes declarei como a mão do meu Deus me fora favorável, como também as palavras do rei, que ele me tinha dito. Então, disseram: Levantemo-nos e edifiquemos. E esforçaram as suas mãos para o bem.

\textbf{RA}: E lhes declarei como a boa mão do meu Deus estivera comigo e também as palavras que o rei me falara. Então, disseram: Disponhamo-nos e edifiquemos. E fortaleceram as mãos para a boa obra.

\textbf{KJ}: Then I told them of the hand of my God which was good upon me; as also the king’s words that he had spoken unto me. And they said, Let us rise up and build. So they strengthened their hands for this good work.

\subsection*{Ne 2.20:} 
 \addcontentsline{toc}{subsection}{2.20}
\begin{quote}
    \small
Então lhes respondi, e disse: O Deus dos céus é o que nos fará prosperar\uline{:} e nós, seus servos, nos levantaremos e edificaremos; mas vós não tendes parte, nem justiça, nem memória em Jerusalém.
\end{quote}

Em vez de dois pontos, ponto-e-vírgula.

\textbf{RC}: Então, lhes respondi e disse: O Deus dos céus é o que nos fará prosperar; e nós, seus servos, nos levantaremos e edificaremos; mas vós não tendes parte, nem justiça, nem memória em Jerusalém.

\textbf{RA}: Então, lhes respondi: o Deus dos céus é quem nos dará bom êxito; nós, seus servos, nos disporemos e reedificaremos; vós, todavia, não tendes parte, nem direito, nem memorial em Jerusalém.

\textbf{KJ}: Then answered I them, and said unto them, The God of heaven, he will prosper us; therefore we his servants will arise and build: but ye have no portion, nor right, nor memorial, in Jerusalem.

\subsection*{Ne 3.12:} 
 \addcontentsline{toc}{subsection}{3.12}
\begin{quote}
    \small
E ao seu lado reparou \uwave{Sallum}, filho de Haloés, líder da outra meia parte de Jerusalém, ele e suas filhas.
\end{quote}

Não seria Salum, com apenas um ``l’’? (cf.~ICr~3.15).

\textbf{RC}: E, ao seu lado, reparou Salum, filho de Haloés, maioral da outra meia parte de Jerusalém, ele e suas filhas.

\textbf{RA}: Ao lado dele, reparou Salum, filho de Haloés, maioral da outra meia parte de Jerusalém, ele e suas filhas.

\textbf{KJ}: And next unto him repaired Shallum the son of Halohesh, the ruler of the half part of Jerusalem, he and his daughters.

\subsection*{Ne 6.18:} 
 \addcontentsline{toc}{subsection}{6.18}
\begin{quote}
    \small
Porque muitos em Judá lhe eram ajuramentados, porque era genro de Secanias~[,] filho de Ará; e seu filho Joanã se casara com a filha de Mesulão, filho de Berequias.
\end{quote}

Acrescentar vírgula.

\textbf{RC}: Porque muitos em Judá se lhe ajuramentaram, porque era genro de Secanias, filho de Ará; e seu filho Joanã tomara a filha de Mesulão, filho de Berequias.

\textbf{RA}: Pois muitos em Judá lhe eram ajuramentados porque era genro de Secanias, filho de Ará; e seu filho Joanã se casara com a filha de Mesulão, filho de Berequias.

\textbf{KJ}: For there were many in Judah sworn unto him, because he was the son in law of Shechaniah the son of Arah; and his son Johanan had taken the daughter of Meshullam the son of Berechiah.

\subsection*{Ne 7.8,11,12,17,34,38,66,67,69:} 
\addcontentsline{toc}{subsection}{7.8,11,12,17,34,38,66,67,69}
\begin{quote}
    \small
$^{\mathrm{8}}$Foram os filhos de Parós, dois mil, cento e setenta e dois. $^{\mathrm{11}}$Os filhos de Paate-Moabe, dos filhos de Jesuá e de Joabe, dois mil, oitocentos e dezoito. $^{\mathrm{12}}$Os filhos de Elão, mil, duzentos e cinqüenta e quatro. $^{\mathrm{17}}$Os filhos de Azgade, dois mil, trezentos e vinte e dois. $^{\mathrm{34}}$Os filhos do outro Elão, mil, duzentos e cinqüenta e quatro:
$^{\mathrm{38}}$Os filhos de Senaá, três mil, novecentos e trinta. $^{\mathrm{66}}$Toda esta congregação junta foi de quarenta e dois mil, trezentos e sessenta, $^{\mathrm{67}}$afora os seus servos e as suas servas, que foram sete mil, trezentos e trinta e sete; e tinham duzentos e quarenta e cinco cantores e cantoras. $^{\mathrm{69}}$Camelos, quatrocentos e trinta e cinco; jumentos, seis
mil, setecentos e vinte.
\end{quote}

Em todos esses versículos, a correta grafia da numeração por extenso:
não há vírgula após o milhar. Por exemplo: \emph{dois mil cento e
 setenta e dois} e não \emph{dois mil, cento e setenta e dois}.

\subsection*{Ne 12.44:} 
\addcontentsline{toc}{subsection}{12.44}
\begin{quote}
    \small
Também no mesmo dia se nomearam homens sobre as câmaras\uline{,} dos tesouros, das ofertas alçadas, das primícias, dos dízimos, para ajuntarem nelas, dos campos das cidades, as partes da lei para os sacerdotes e para os levitas; porque Judá estava alegre por causa dos sacerdotes e dos levitas que assistiam ali.
\end{quote}

Não há essa vírgula após ``câmaras’’. É ``sobre as câmaras dos tesouros’’.

RC: Também, no mesmo dia, se nomearam homens sobre as câmaras, para os tesouros, para as ofertas alçadas, para as primícias e para os dízimos, para ajuntarem nelas, das terras das cidades, as porções designadas pela Lei para os sacerdotes e para os levitas; porque Judá estava alegre por causa dos sacerdotes e dos levitas que assistiam ali.

RA: Ainda no mesmo dia, se nomearam homens para as câmaras dos tesouros, das ofertas, das primícias e dos dízimos, para ajuntarem nelas, das cidades, as porções designadas pela Lei para os sacerdotes e para os levitas; pois Judá estava alegre, porque os sacerdotes e os levitas ministravam ali;

King James: And at that time were some appointed over the chambers for the treasures, for the offerings, for the firstfruits, and for the tithes, to gather into them out of the fields of the cities the portions of the law for the priests and Levites: for Judah rejoiced for the priests and for the Levites that waited.

\subsection*{Ne 13.31:} 
 \addcontentsline{toc}{subsection}{13.31} 
 \begin{quote}
    \small
Como também para com as ofertas de lenha em tempos determinados, e para com as primícias\uline{;} lembra-te de mim, Deus meu, para bem.
\end{quote}

Em vez de ponto-e-vírgula, ponto.

RC: Como também para as ofertas da lenha em tempos determinados e para as primícias. Lembra-te de mim, Deus meu, para o bem.

RA: Como também o fornecimento de lenha em tempos determinados, bem como as primícias. Lembra-te de mim, Deus meu, para o meu bem.

King James: And for the wood offering, at times appointed, and for the firstfruits. Remember me, O my God, for good.

\section{Ester}
\subsection*{Et 8.10:} 
 \addcontentsline{toc}{subsection}{8.10} 
 \begin{quote}
    \small
E escreveu-se em nome do rei Assuero e, selando-as com o anel do rei, enviaram as cartas pela mão de correios a cavalo, que cavalgavam sobre ginetes, que eram das cavalariças do rei.
\end{quote}

Aqui uma questão interessante: por que a King James faz referência a
mulas, camelos e dromedários e as traduções para o português os
ignoram? Afinal, o Texto Massorético contém ou não essa citação?

King James: And he wrote in the king Ahasuerus’ name, and sealed it with the king’s ring, and sent letters by posts \uwave{on horseback, and riders on mules, camels, and young dromedaries.}

RC: E se escreveu em nome do rei Assuero, e se selou com o anel do rei; e se enviaram as cartas pela mão de correios a cavalo e que cavalgavam sobre ginetes, que eram das cavalariças do rei.

RA: Escreveu-se em nome do rei Assuero, e se selou com o anel do rei; as cartas foram enviadas por intermédio de correios montados em ginetes criados na coudelaria do rei.

\chapter{Livros Sapienciais}
\section{Livro de Jó}
\subsection*{Jó 1.5:} 
 \addcontentsline{toc}{subsection}{1.5} 
 \begin{quote}
    \small
Sucedia, pois, que, decorrido o turno de dias de seus banquetes, \uwave{enviava} Jó, e os santificava, e se levantava de madrugada, e oferecia holocaustos segundo o número de todos eles; porque dizia Jó: Talvez pecaram meus filhos, e amaldiçoaram a Deus no seu coração. Assim fazia Jó continuamente.
\end{quote}

Enviava o quê?

\textbf{RA}: Decorrido o turno de dias de seus banquetes, chamava Jó a seus filhos e os santificava; levantava-se de madrugada e oferecia holocaustos segundo o número de todos eles, pois dizia: Talvez tenham pecado os meus filhos e blasfemado contra Deus em seu coração. Assim o fazia Jó continuamente.

\textbf{KJ}: And it was so, when the days of their feasting were gone about, that Job sent and sanctified them, and rose up early in the morning, and offered burnt offerings according to the number of them all: for Job said, It may be that my sons have sinned, and cursed God in their hearts. Thus did Job continually.

\subsection*{Jó 5.25-26:} 
 \addcontentsline{toc}{subsection}{5.25-26} 
 \begin{quote}
    \small
 Também saberás que se multiplicará a tua descendência e a tua posteridade como a erva da terra\uwave{,} $^{\mathrm{26}}$na velhice irás à sepultura, como se recolhe o feixe de trigo a seu tempo.
\end{quote}

\textbf{KJ}: Thou shalt know also that thy seed shall be great, and thine offspring as the grass of the earth. 26Thou shalt come to thy grave in a full age, like as a shock of corn cometh in in his season.

\textbf{RA}: 25 Saberás também que se multiplicará a tua descendência, e a tua posteridade, como a erva da terra. 26 Em robusta velhice entrarás para a sepultura, como se recolhe o feixe de trigo a seu tempo.

\textbf{RC}: 25Também saberás que se multiplicará a tua semente, e a tua posteridade, como a erva da terra. 26 Na velhice virás à sepultura, como se recolhe o feixe de trigo a seu tempo.

\subsection*{Jó 6.3:} 
 \addcontentsline{toc}{subsection}{6.3}
 \begin{quote}
    \small
 Porque, na verdade, mais pesada seria\uwave{,} do que a areia dos mares; por isso é que as minhas palavras têm sido engolidas.
\end{quote}

Não há essa vírgula.

\textbf{RC}: Porque, na verdade, mais pesada seria do que a areia dos mares; por isso é que as minhas palavras têm sido inconsideradas.

\textbf{KJ}: For now it would be heavier than the sand of the sea: therefore my words are swallowed up.

\subsection*{Jó 6.15-17:} 
 \addcontentsline{toc}{subsection}{6.15-17} 
 \begin{quote}
    \small
Meus irmãos aleivosamente me trataram, como um ribeiro, como a torrente dos ribeiros que passam, $^{\mathrm{16}}$que estão encobertos com a geada, e neles se esconde a neve\uwave{,} $^{\mathrm{17}}$no tempo em que se derretem com o calor, se desfazem, e em se aquentando, desaparecem do seu lugar.
\end{quote}

Conferir pontuação.

\textbf{RC}: 15 Meus irmãos aleivosamente me trataram; são como um ribeiro, como a torrente dos ribeiros que passam, 16 que estão encobertos com a geada, e neles se esconde a neve. 17No tempo em que se derretem com o calor, se desfazem; e, em se aquentando, desaparecem do seu lugar.

\textbf{KJ}: 15 My brethren have dealt deceitfully as a brook, and as the stream of brooks they pass away; 16which are blackish by reason of the ice, and wherein the snow is hid: 17What time they wax warm, they vanish: when it is hot, they are consumed out of their place.

\textbf{RA}: 15Meus irmãos aleivosamente me trataram; são como um ribeiro, como a torrente que transborda no vale, 16turvada com o gelo e com a neve que nela se esconde, 17torrente que no tempo do calor seca, emudece e desaparece do seu lugar.

\subsection*{Jó 10.17:} 
 \addcontentsline{toc}{subsection}{10.17}
 \begin{quote}
    \small
 Tu renovas contra mim as tuas testemunhas, e multiplicas contra mim a tua ira; \uwave{revezes} e combate estão comigo.
\end{quote}

O plural de revés é REVESES: Reverso. Golpe aplicado com as costas da mão. Pancada oblíqua. Acidente desfavorável; vicissitude. Fig. Desgraça, infortúnio, insucesso. REVEZES: Usado nas loc. a revezes e às revezes. A revezes: uma vez ou outra; às vezes, por vezes, de vez em quando; alternativamente; às revezes.

\textbf{RC}: Tu renovas contra mim as tuas testemunhas e multiplicas contra mim a tua ira; reveses e combate estão comigo.

\textbf{RA}: Tu renovas contra mim as tuas testemunhas e multiplicas contra mim a tua ira; males e lutas se sucedem contra mim.

\textbf{KJ}: Thou renewest thy witnesses against me, and increasest thine indignation upon me; changes and war are against me.

\subsection*{Jó 11.11:} 
 \addcontentsline{toc}{subsection}{11.11}
 \begin{quote}
    \small
 Porque ele conhece \uwave{aos homens} vãos, e vê o vício; e não o terá em consideração?
\end{quote}

\textbf{RC}: Porque ele conhece os homens vãos e vê o vício; e não o terá em consideração?

\textbf{KJ}: For he knoweth vain men: he seeth wickedness also; will he not then consider it?

\textbf{RA}: Porque ele conhece os homens vãos e, sem esforço, vê a iniqüidade.

\subsection*{Jó 13.21:} 
 \addcontentsline{toc}{subsection}{13.21}
 \begin{quote}
    \small
\ldots desvia a tua mão para longe\uwave{, de mim,} e não me espante o teu terror.
\end{quote}

Não há essas vírgulas.

\textbf{RC}: Desvia a tua mão para longe de mim e não me espante o teu terror.

\textbf{KJ}: Withdraw thine hand far from me: and let not thy dread make me afraid.

\subsection*{Jó 16.16-17:} 
\addcontentsline{toc}{subsection}{16.16-17}
 \begin{quote}
    \small
O meu rosto está todo avermelhado de chorar, e sobre as minhas pálpebras está a sombra da morte\uwave{:} $^{\mathrm{17}}$Apesar de não haver violência nas minhas mãos, e de ser pura a minha oração.
\end{quote}

Vírgula ou ponto-e-vírgula, mas não dois pontos.

\textbf{RC}: 16 O meu rosto todo está descorado de chorar, e sobre as minhas pálpebras está a sombra da morte, 17apesar de não haver violência nas minhas mãos e de ser pura a minha oração.

\textbf{KJ}: 16My face is foul with weeping, and on my eyelids is the shadow of death; 17not for any injustice in mine hands: also my prayer is pure.

\textbf{RA}: 16O meu rosto está todo afogueado de chorar, e sobre as minhas pálpebras está a sombra da morte, 17embora não haja violência nas minhas mãos, e seja pura a minha oração.

\subsection*{Jó 17.11:} 
 \addcontentsline{toc}{subsection}{17.11}
 \begin{quote}
    \small
Os meus dias passaram, e \uwave{malograram} os meus propósitos, as aspirações do meu coração.
\end{quote}

Ou malograram-se?

\textbf{RC}: Os meus dias passaram, e malograram-se os meus propósitos, as aspirações do meu coração.

\textbf{KJ}: My days are past, my purposes are broken off, even the thoughts of my heart.

\textbf{RA}: Os meus dias passaram, e se malograram os meus propósitos, as aspirações do meu coração.

\subsection*{Jó 20.6-7:} 
 \addcontentsline{toc}{subsection}{20.6-7}
 \begin{quote}
    \small
Ainda que a sua altivez suba até ao céu, e a sua cabeça chegue até às nuvens\uwave{.} Contudo, como o seu próprio esterco, perecerá para sempre; e os que o viam dirão: Onde está?
\end{quote}

Vírgula ou ponto-e-vírgula, mas não ponto.

\textbf{RC}: 6Ainda que a sua altura suba até ao céu, e a sua cabeça chegue até às nuvens, como o seu próprio esterco perecerá para sempre; e os que o viam dirão: Onde está?

\textbf{KJ}: Though his excellency mount up to the heavens, and his head reach unto the clouds; yet he shall perish for ever like his own dung: they which have seen him shall say, Where is he?

\textbf{RA}: Ainda que a sua presunção remonte aos céus, e a sua cabeça atinja as nuvens, como o seu próprio esterco, apodrecerá para sempre; e os que o conheceram dirão: Onde está?

\subsection*{Jó 26.13:} 
 \addcontentsline{toc}{subsection}{26.13}
 \begin{quote}
    \small
Pelo seu \uwave{Espírito} ornou os céus; a sua mão formou a serpente \uwave{enroscadiça}.
\end{quote}

Conferir ``Espírito’’ ou ``espírito’’. Ainda: ``enroscadiça’’, termo não encontrado no dicionário.

\textbf{KJ}: By his spirit he hath garnished the heavens; his hand hath formed the crooked serpent.

\textbf{RA}: Pelo seu sopro aclara os céus, a sua mão fere o dragão veloz.

\subsection*{Jó 27.2-4:} 
 \addcontentsline{toc}{subsection}{27.2-4}
 \begin{quote}
    \small
Vive Deus, que desviou a minha causa, e o Todo-Poderoso, que amargurou a minha alma\uwave{.} $^{\mathrm{3}}$Que, enquanto em mim houver alento, e o sopro de Deus nas minhas narinas, $^{\mathrm{4}}$não falarão os meus lábios iniqüidade, nem a minha língua pronunciará engano.
 \end{quote}
 
Conferir esse ponto.

\textbf{RC}: 2 Vive Deus, que desviou a minha causa, e o Todo-poderoso, que amargurou a minha alma. 3Enquanto em mim houver alento, e o sopro de Deus no meu nariz,

\textbf{KJ}: 2As God liveth, who hath taken away my judgment; and the Almighty, who hath vexed my soul; 3all the while my breath is in me, and the spirit of God is in my nostrils;

\textbf{RA}: Tão certo como vive Deus, que me tirou o direito, e o Todo-Poderoso, que amargurou a minha alma, 3 enquanto em mim estiver a minha vida, e o sopro de Deus nos meus narizes,

\subsection*{Jó 30.26:} 
 \addcontentsline{toc}{subsection}{30.26}
 \begin{quote}
    \small
Todavia aguardando eu o bem, então me veio o mal\uwave{,} esperando eu a luz, veio a escuridão.
\end{quote}

Em vez de vírgula, ponto-e-vírgula após ``mal’’.

\textbf{RC}: Todavia, aguardando eu o bem, eis que me veio o mal; e, esperando eu a luz, veio a escuridão.

\textbf{KJ}: When I looked for good, then evil came unto me: and when I waited for light, there came darkness.

\textbf{RA}: Aguardava eu o bem, e eis que me veio o mal; esperava a luz, veio-me a escuridão.

\subsection*{Jó 31.16-22:} 
\addcontentsline{toc}{subsection}{31.16-22}
 \begin{quote}
    \small
Se retive o que os pobres desejavam, ou fiz desfalecer os olhos da viúva, $^{\mathrm{17}}$ou se\uwave{,} sozinho comi o meu bocado, e o órfão não comeu dele $^{\mathrm{18}}$(porque desde a minha mocidade cresceu comigo como com seu pai, e fui o guia da viúva desde o ventre de minha mãe), $^{\mathrm{19}}$se alguém vi perecer por falta de roupa, e ao necessitado por não ter coberta, $^{\mathrm{20}}$se os seus lombos não me abençoaram, se ele não se aquentava com as peles dos meus cordeiros, $^{\mathrm{21}}$se eu levantei a minha mão contra o órfão, porquanto na porta via a minha ajuda, $^{\mathrm{22}}$então caia do ombro a minha espádua, e separe-se o meu braço do osso.
\end{quote}

\textbf{RC}: 16Se retive o que os pobres desejavam ou fiz desfalecer os olhos da viúva; 17ou sozinho comi o meu bocado, e o órfão não comeu dele 18(porque desde a minha mocidade cresceu comigo como com seu pai, e o guiei desde o ventre da minha mãe); 19se a alguém vi perecer por falta de veste e, ao necessitado, por não ter coberta; 20se os seus lombos me não abençoaram, se ele não se aquentava com as peles dos meus cordeiros; 21se eu levantei a mão contra o órfão, porque na porta via a minha ajuda, 22então, caia do ombro a minha espádua, e quebre-se o meu braço desde o osso.

\textbf{KJ}: 16If I have withheld the poor from their desire, or have caused the eyes of the widow to fail; 17or have eaten my morsel myself alone, and the fatherless hath not eaten thereof; 18(for from my youth he was brought up with me, as with a father, and I have guided her from my mother’s womb;) 19if I have seen any perish for want of clothing, or any poor without covering; 20if his loins have not blessed me, and if he were not warmed with the fleece of my sheep; 21if I have lifted up my hand against the fatherless, when I saw my help in the gate: 22then let mine arm fall from my shoulder blade, and mine arm be broken from the bone.

\subsection*{Jó 33.12:} 
 \addcontentsline{toc}{subsection}{33.12}
 \begin{quote}
    \small
Eis que nisso não tens razão\uwave{;} eu te respondo; porque maior é Deus do que o homem.
\end{quote}

\textbf{RC}: Eis que nisto te respondo: Não foste justo; porque maior é Deus do que o homem.

\textbf{KJ}: Behold, in this thou art not just: I will answer thee, that God is greater than man.

\textbf{RA}: Nisto não tens razão, eu te respondo; porque Deus é maior do que o homem.

\subsection*{Jó 36.4:} 
\addcontentsline{toc}{subsection}{36.4}
 \begin{quote}
    \small
Porque [,] na verdade, as minhas palavras não serão falsas; contigo está um que tem perfeito conhecimento.
\end{quote}

Acrescentar a vírgula indicada.

\textbf{RC}: Porque, na verdade, as minhas palavras não serão falsas; contigo está um que é sincero na sua opinião.

\textbf{KJ}: For truly my words shall not be false: he that is perfect in knowledge is with thee.

\textbf{RA}: Porque, na verdade, as minhas palavras não são falsas; contigo está quem é senhor do assunto.

\subsection*{Jó 37.14:} 
 \addcontentsline{toc}{subsection}{37.14}
 \begin{quote}
    \small
A isto, ó Jó, inclina os teus ouvidos; \uwave{para}, e considera as maravilhas de Deus.
\end{quote}

``Pára, e considera \ldots’’. Faltou o acento.

\textbf{RC}: A isto, ó Jó, inclina os teus ouvidos; atende e considera as maravilhas de Deus.

\textbf{KJ}: Hearken unto this, O Job: stand still, and consider the wondrous works of God.

\textbf{RA}: Inclina, Jó, os ouvidos a isto, pára e considera as maravilhas de Deus.

\subsection*{Jó 39.30:} 
 \addcontentsline{toc}{subsection}{39.30}
 \begin{quote}
    \small
E seus filhos chupam o sangue\uwave{,} e onde há mortos, ali está ela.
Em vez de vírgula, ponto-e-vírgula. 
\end{quote}

\textbf{RC}: Seus filhos chupam o sangue; e onde há mortos, ela aí está.

\textbf{KJ}: Her young ones also suck up blood: and where the slain are, there is she.

\textbf{RA}: Seus filhos chupam sangue; onde há mortos, ela aí está.

\subsection*{Sl 12.6-7} 
 \addcontentsline{toc}{subsection}{12.6-7}
 \begin{quote}
    \small
As palavras do SENHOR são palavras puras, como prata refinada em fornalha de barro, purificada sete vezes. $^{\mathrm{7}}$Tu OS guardarás, SENHOR; desta geração OS livrarás para sempre.
\end{quote}

Cabe aqui a sugestão do irmão Hélio de Menezes Silva:

``As palavras de Jeová são palavras puras, como prata refinada em fornalha de barro, purificada sete vezes. $^{\mathrm{7}}$Tu AS guardarás, Jeová; desta geração AS livrarás para sempre.

À primeira vista, ``as’’ (referindo-se às palavras de Deus)) e ``os’’
`(referindo-se aos crentes fiéis) são, ambas, traduções possíveis. Mas a
tradução ``as’’ provavelmente é melhor, pois a gramática nos ensina que
pronomes (``as’’) usualmente se referem ao mais próximo antecedente que
lhes casa (``palavras’’)!

De qualquer modo, até mesmo por segurança, temos que crer AMBAS as doces alternativas! (louvado seja Deus por elas!).’’

Não descartemos, não joguemos no lixo nenhuma delas! Deus preserva perfeitamente seus filhos fiéis, e Sua Palavra infalível.

\subsection*{Sl 26.6-7:} 
 \addcontentsline{toc}{subsection}{26.6-7}
 \begin{quote}
    \small
Lavo as minhas mãos na inocência; e assim andarei, Senhor, ao redor do teu altar\uwave{.} $^{\mathrm{7}}$Para publicar com voz de louvor, e contar todas as tuas maravilhas.
\end{quote}

Ver outras versões. 

\subsection*{Sl 41.6:} 
 \addcontentsline{toc}{subsection}{41.6}
 \begin{quote}
    \small
 E, se algum deles vem ver-me, fala coisas vãs; no seu coração amontoa a maldade; \uwave{saindo para fora}, é disso que fala.
 \end{quote}
 
Pleonasmo vicioso (\ref{saiu}).

\subsection*{Sl 41.13:} 
 \addcontentsline{toc}{subsection}{41.13}
 \begin{quote}
    \small
Bendito seja o Senhor Deus de Israel de século em século. Amém e Amém.
Deus de Israel não seria entre vírgulas? Ver outras versões. 
\end{quote}

\subsection*{Sl 50.21-22:} 
 \addcontentsline{toc}{subsection}{50.21-22}
 \begin{quote}
    \small
Estas coisas tens feito, e eu me calei; pensavas que era tal como tu, mas eu te argüirei, e as porei por ordem diante dos teus olhos\uwave{:} $^{\mathrm{22}}$Ouvi pois isto, vós que vos esqueceis de Deus; para que eu vos não faça em pedaços, sem haver quem vos livre.
\end{quote}

Não seria simplesmente um ponto em vez de dois pontos? Ver outras versões. 

\subsection*{Sl 68.9-10:} 
 \addcontentsline{toc}{subsection}{68.9-10}
 \begin{quote}
    \small
Tu, ó Deus, mandaste a chuva em abundância, confortaste a tua herança, quando estava cansada. $^{\mathrm{10}}$Nela habitava o teu rebanho\uwave{;} tu, ó Deus, fizeste provisão da tua bondade para o pobre.
\end{quote}

Conferir ponto-e-vírgula. Ver outras versões. 

\subsection*{Sl 94.10:} 
 \addcontentsline{toc}{subsection}{94.10 }
 \begin{quote}
    \small
Aquele que \uwave{argüi} os gentios não castigará? E o que ensina ao homem o conhecimento, não saberá.
\end{quote}

Argúi.

\subsection*{Sl 124.3:} 
 \addcontentsline{toc}{subsection}{124.3}
 \begin{quote}
    \small
 \ldots eles então nos teriam \uwave{engulido} vivos, quando a sua ira se acendeu contra nós.
\end{quote}

Engolido.

\subsection*{Sl 138.2} 
\addcontentsline{toc}{subsection}{138.2}
Querido irmão Hélio:

O título deve ter deixado você curioso e ao mesmo tempo ansioso. É que acabei de fazer um achado fantástico! Tudo começou quando li o primeiro email de inauguração do nosso grupo, e este trazia o seguinte versículo:
 \begin{quote}
    \small
Salmos 138:2 - Inclinar-me-ei para o teu santo templo, e louvarei o teu nome pela tua benignidade, e pela tua verdade; pois engrandeceste a tua palavra acima de todo o teu nome. (ACF - Almeida Corrigida Fiel)
\end{quote}

É que eu já havia lido este versículo numa Bíblia ``alexandrina’’ - Almeida Atualizada - e reparei uma destoante diferença na tradução:

\textbf{RA}: Prostrar-me-ei para o teu santo templo e louvarei o teu nome, por causa da tua misericórdia e da tua verdade, pois magnificaste acima de tudo o teu nome e a tua palavra. (ARA - Almeida Revista e Atualizada)

Notou a diferença? Na primeira Deus engrandeceu a Sua palavra acima de todo o Seu nome; na segunda, Deus engrandeceu/magnificou o Seu nome e a Sua palavra acima de tudo!!! Daí pensei: Este é mais um daqueles versículos que trazem variantes. E fui verificar na minha Bíblia hebraica de onde procedia aquela segunda variante da ARA. Fiquei estarrecido. A tradução da ARA está baseada em NENHUM MANUSCRITO!!! Percebeu? Este é o novo manuscrito que achei, e decidi nomeá-lo de NENHUM, já que ele não existe!!! Isso mesmo ele não existe, e vou explicar o por quê.

Abrindo um comentário dos Salmos, de autoria de Derek Kidner - Ed. Vida Nova - encontrei o seguinte comentário acerca do já citado verso:

- ``O texto hebraico, conforme o temos, se traduz em AV [Authorized Version], RV [Revised Version], ARC [Almeida Revista e Corrigida]: ``pois engrandeceste a tua palavra acima de todo o teu nome’’. É uma expressão estranha, e uma declaração estranha se ``teu nome’’ tem o significado usual da revelação da própria Pessoa de Deus, conforme tem na primeira metade do versículo... parece justificada em supor que um copista omitiu a letra w, com o significado de ``e’’, de um texto que dizia: pois magnificaste acima de tudo o teu nome e a tua palavra.’’ ( grifos meus ).

Ora, os tradutores moderninhos sentem-se à vontade para supor que um copista omitiu isso ou aquilo, e ao seu bel-prazer modificam o Texto Massorético baseados numa suposição sugerida por um ``erudito’’ qualquer. O texto único deste versículo em hebraico é o da ACF, ARC, KJV, e tantos outros que fielmente traduziram o que está no texto original, sem se preocupar se o versículo vai ficar com ``uma expressão estranha’’. Repito, este texto NÃO traz variantes no original, e portanto a tradução da Almeida Revista e Atualizada está baseada em NENHUM MANUSCRITO!!! O texto é de uma interpretação difícil, sim. Mas é a ÚNICA possível baseada nos manuscritos que temos. A BÍBLIA HEBRAICA STUTTGARTENSIA - de onde é traduzida a maioria das Bíblias ditas modernas, apesar de trazer o texto que traduzido formalmente quer dizer: pois magnificaste acima de tudo o teu nome e a tua palavra (ACF), traz no aparato crítico ao pé da página, uma nota onde diz que o texto provável é o da ARA!!!???. Como já disse, esta tradução da ARA está baseada em parece... supor... provável. Desafio qualquer um citar um manuscrito sequer que dê suporte à invenção da ARA. Este manuscrito NÃO existe. Daí eu ter achado o famoso MANUSCRITO NENHUM. Pediria aos senhores ``eruditos’’ que tivessem um mínimo de respeito pela Palavra de Deus. E ao invés de tentar corrigir o que Deus disse, humildemente dessem a mão à palmatória e traduzissem o que o texto sagrado traz.

Irmão, precisamos estar de olhos bem abertos nestes dias LAUDICENIANOS!!!

Shalom, Euclides

\subsection*{Sl 142.7:} 
 \addcontentsline{toc}{subsection}{142.7}
 \begin{quote}
    \small
Tira a minha alma da prisão, para que louve o teu nome\uwave{;} os justos me rodearão, pois me fizeste bem.
\end{quote}

Ver outras versões se não poderia ser um ponto.

\subsection*{Sl 145.3:} 
 \addcontentsline{toc}{subsection}{145.3}
 \begin{quote}
    \small
Grande é o Senhor, e muito digno de louvor, e a sua grandeza\uwave{inexcrutável}.
\end{quote}

\textit{\textbf{IneScrutável}}.

\subsection*{Sl 146.7b-10:} 
 \addcontentsline{toc}{subsection}{146.7b-10}
 \begin{quote}
    \small
 O Senhor solta os encarcerados. $^{\mathrm{8}}$O Senhor abre os olhos aos cegos; o Senhor levanta os abatidos; o Senhor ama os justos; $^{\mathrm{9}}$o Senhor guarda os estrangeiros; sustém o órfão e a viúva, mas transtorna o caminho dos ímpios. $^{\mathrm{10}}$O Senhor reinará eternamente; o teu Deus, ó Sião, de geração em geração. Louvai ao Senhor.
\end{quote}

Critério para a utilização de pontos ou ponto-e-vírgulas.

\section{Provérbios}
\subsection*{Pv 1.2:} 
 \addcontentsline{toc}{subsection}{1.2}
 \begin{quote}
    \small
 \ldots para se conhecer a sabedoria e a instrução; para se entenderem\uwave{,} as palavras da prudência. \ldots para dar aos simples, prudência, e aos moços, conhecimento e bom siso\uwave{;} $^{\mathrm{5}}$o sábio ouvirá e crescerá em conhecimento, e o entendido adquirirá sábios conselhos; \ldots
\end{quote}

\subsection*{Pv 1.29-30:} 
 \addcontentsline{toc}{subsection}{1.29-30:}
\begin{quote} 
 \small 
Porquanto odiaram o conhecimento; e não preferiram o temor do Senhor\uwave{:} $^{\mathrm{30}}$Não aceitaram o meu conselho, e desprezaram toda a minha repreensão.
\end{quote}

\subsection*{Pv 12.19:} 
 \addcontentsline{toc}{subsection}{12.19}
 \begin{quote} 
 \small 
 O lábio da verdade permanece para sempre, mas a língua da falsidade\uwave{,} dura por um só momento.
\end{quote}

\subsection*{Pv 23.1-2:} 
 \addcontentsline{toc}{subsection}{23.1-2}
 \begin{quote} 
 \small
 QUANDO te assentares a comer com um governador, atenta bem para o que é posto diante de ti\uwave{,} $^{\mathrm{2}}$e se és homem de grande apetite, põe uma faca à tua garganta.
\end{quote}

\subsection*{Pv 24.23:} 
 \addcontentsline{toc}{subsection}{24.23}
 \begin{quote} 
 Também estes são provérbios dos sábios: \uwave{Ter respeito a pessoas no julgamento não é bom}.
 \end{quote}
 
Esclarecer o sentido.

\subsection*{Pv 25.23:} 
 \addcontentsline{toc}{subsection}{25.23}
 \begin{quote} 
 O vento norte afugenta a chuva\uwave{,} e a face irada\uwave{,} a língua fingida.
\end{quote}

\subsection*{Pv 30.24-30:} 
 \addcontentsline{toc}{subsection}{30.24-30}
 \begin{quote} 
 Estas quatro coisas são das menores da terra, porém bem providas de sabedoria: $^{\mathrm{25}}$as formigas não são um povo forte\uwave{;} todavia no verão preparam a sua comida; $^{\mathrm{26}}$os coelhos são um povo débil\uwave{;} e contudo, põem a sua casa na rocha; $^{\mathrm{27}}$os gafanhotos não têm rei\uwave{;} e contudo todos saem, e em bandos se repartem; $^{\mathrm{28}}$a aranha se pendura com as mãos, e está nos palácios dos reis. $^{\mathrm{29}}$Estes três têm um bom andar, e quatro passeiam airosamente\uwave{;} $^{\mathrm{30}}$o leão, o mais forte entre os animais, que não foge de nada; 31o galo; o bode também; e o rei a quem não se pode resistir.
\end{quote}

\section{Eclesiastes}
\subsection*{Ec 1.3:} 
 \addcontentsline{toc}{subsection}{1.3}
 \begin{quote} 
 Que proveito tem o homem, de todo o seu trabalho, que faz debaixo do sol.
\end{quote}

Há vírgula depois de `homem’ ou de `trabalho’?

\textbf{RC}: Que vantagem tem o homem de todo o seu trabalho, que ele faz debaixo do sol?

\textbf{RA}: Que proveito tem o homem de todo o seu trabalho, com que se afadiga debaixo do sol?

KJ: What profit hath a man of all his labour which he taketh under the sun?

\subsection*{Ec 3.18:} 
 \addcontentsline{toc}{subsection}{3.18}
 \begin{quote} 
 Disse eu no meu coração, \uwave{quanto a condição} dos filhos dos homens, que Deus os provaria, para que assim pudessem ver que são em si mesmos como os animais.
\end{quote}

Há crase: ``quanto à condição’’.

\subsection*{Ec 3.22:} 
 \addcontentsline{toc}{subsection}{3.22}
 \begin{quote} 
 Assim que tenho visto que não há coisa melhor do que \uwave{alegrar-se o homem} nas suas obras, porque essa é a sua porção; pois quem o fará voltar para ver o que será depois dele?
\end{quote}

 Em vez de ``do que alegrar-se o homem’’, seria ``do que se alegrar o homem’’. O ``que’’ atrai o ``se’’.

\subsection*{Ec 8.9:} 
 \addcontentsline{toc}{subsection}{8.9}
 \begin{quote} 
 Tudo isto vi quando apliquei o meu coração a toda a obra que se faz debaixo do sol\uwave{;} tempo há em que um homem tem domínio sobre outro homem, para desgraça sua.
 \end{quote}
 
Ponto-e-vírgula? Não seria melhor dois pontos?

\textbf{RC}: Tudo isso vi quando apliquei o meu coração a toda obra que se faz debaixo do sol; tempo há em que um homem tem domínio sobre outro homem, para desgraça sua.

\textbf{KJ}: All this have I seen, and applied my heart unto every work that is done under the sun: there is a time wherein one man ruleth over another to his own hurt.

\textbf{RA}: Tudo isto vi quando me apliquei a toda obra que se faz debaixo do sol; há tempo em que um homem tem domínio sobre outro homem, para arruiná-lo.

\subsection*{Ec 8.16-17:} 
 \addcontentsline{toc}{subsection}{8.16-17}
 \begin{quote} 
 Aplicando eu o meu coração a conhecer a sabedoria, e a ver o trabalho que há sobre a terra (que nem de dia nem de noite vê o homem sono nos seus olhos); $^{\mathrm{17}}$então vi toda a obra de Deus, que o homem não pode perceber, a obra que se faz debaixo do sol, por mais que trabalhe o homem para a descobrir, não a achará; e, ainda que diga o sábio que a conhece, nem por isso a poderá compreender.
 \end{quote}

Truncado

\textbf{RC}: Aplicando eu o meu coração a conhecer a sabedoria e a ver o trabalho que há sobre a terra (pois nem de dia nem de noite vê o homem sono nos seus olhos), então, vi toda a obra de Deus, que o homem não pode alcançar a obra que se faz debaixo do sol; por mais que trabalhe o homem para a buscar, não a achará; e, ainda que diga o sábio que a virá a conhecer, nem por isso a poderá alcançar.

RA: Aplicando-me a conhecer a sabedoria e a ver o trabalho que há sobre a terrappois nem de dia nem de noite vê o homem sono nos seus olhos,, então, contemplei toda a obra de Deus e vi que o homem não pode compreender a obra que se faz debaixo do sol; por mais que trabalhe o homem para a descobrir, não a entenderá; e, ainda que diga o sábio que a virá a conhecer, nem por isso a poderá achar.

\textbf{KJ}: When I applied mine heart to know wisdom, and to see the business that is done upon the earth: (for also there is that neither day nor night seeth sleep with his eyes:) Then I beheld all the work of God, that a man cannot find out the work that is done under the sun: because though a man labour to seek it out, yet he shall not find it; yea further; though a wise man think to know it, yet shall he not be able to find it.

\subsection*{Ec 9.3:} 
 \addcontentsline{toc}{subsection}{9.3}
 \begin{quote} 
 Este é o mal que há entre tudo quanto se faz debaixo do sol\uwave{;} a todos sucede o mesmo; e que também o coração dos filhos dos homens está cheio de maldade, e que há desvarios no seu coração enquanto vivem, e depois se vão aos mortos.
 \end{quote}

Dois pontos em vez de ponto-e-vírgula.

\textbf{RC}: Este é o mal que há entre tudo quanto se faz debaixo do sol: que a todos sucede o mesmo; que também o coração dos filhos dos homens está cheio de maldade; que há desvarios no seu coração, na sua vida, e que depois se vão aos mortos.

\textbf{RA}: Este é o mal que há em tudo quanto se faz debaixo do sol: a todos sucede o mesmo; também o coração dos homens está cheio de maldade, nele há desvarios enquanto vivem; depois, rumo aos mortos.

\textbf{KJ}: This is an evil among all things that are done under the sun, that there is one event unto all: yea, also the heart of the sons of men is full of evil, and madness is in their heart while they live, and after that they go to the dead.

\section{Isaías}
\subsection*{Is 2.19:} 
 \addcontentsline{toc}{subsection}{2.19}
 \begin{quote} 
 Então os homens entrarão nas cavernas das rochas, e nas covas da terra, do terror do Senhor, e da glória da sua majestade, quando ele se levantar para assombrar a terra.
 \end{quote}
 
Tuncado, sem sentido!

\textbf{RC}: Então, os homens se meterão nas concavidades das rochas e nas cavernas da terra, por causa da presença espantosa do SENHOR e por causa da glória da sua majestade, quando ele se levantar para assombrar a terra.

\textbf{RA}: Então, os homens se meterão nas cavernas das rochas e nos buracos da terra, ante o terror do SENHOR e a glória da sua majestade, quando ele se levantar para espantar a terra.

\textbf{KJ}: And they shall go into the holes of the rocks, and into the caves of the earth, for fear of the LORD, and for the glory of his majesty, when he ariseth to shake terribly the earth.

\subsection*{Is 3.16-23:} 
 \addcontentsline{toc}{subsection}{3.16-23}
 \begin{quote} 
 Diz ainda mais o Senhor: Porquanto as filhas de Sião se exaltam, e andam com o pescoço erguido, lançando olhares impudentes; e quando andam, caminham afetadamente, fazendo um tilintar com os seus pés; $^{\mathrm{17}}$portanto o Senhor fará tinhoso o alto da cabeça das filhas de Sião, e o Senhor porá a descoberto a sua nudez\uwave{,} $^{\mathrm{18}}$naquele dia tirará o Senhor os ornamentos dos pés, e as toucas, e adornos em forma de lua, $^{\mathrm{19}}$os pendentes, e os braceletes, as estolas, $^{\mathrm{20}}$os gorros, e os ornamentos das pernas, e os cintos e as caixinhas de perfumes, e os brincos, $^{\mathrm{21}}$os anéis, e as jóias do nariz, $^{\mathrm{22}}$os vestidos de festa, e os mantos, e os xales, e as bolsas\uwave{.} $^{\mathrm{23}}$s espelhos, e o linho finíssimo, e os turbantes, e os véus.
\end{quote}

\textbf{RC}: Diz ainda mais o SENHOR: Porquanto as filhas de Sião se exaltam, e andam de pescoço erguido, e têm olhares impudentes, e, quando andam, como que vão dançando, e cascavelando com os pés, portanto, o Senhor fará tinhosa a cabeça das filhas de Sião e o SENHOR porá a descoberto a sua nudez. Naquele dia, tirará o Senhor o enfeite das ligas, e as redezinhas, e as luetas, e os pendentes, e as manilhas, e as vestes resplandecentes; os diademas, e os enfeites dos braços, e as cadeias, e as caixinhas de perfumes e as arrecadas; os anéis e as jóias pendentes do nariz; as vestes de festa, e os mantos, e as coifas, e os alfinetes; os espelhos, e as capinhas de linho finíssimas, e as toucas, e os véus.

\textbf{RA}: Diz ainda mais o SENHOR: Visto que são altivas as filhas de Sião e andam de pescoço emproado, de olhares impudentes, andam a passos curtos, fazendo tinir os ornamentos de seus pés, o Senhor fará tinhosa a cabeça das filhas de Sião, o SENHOR porá a descoberto as suas vergonhas. Naquele dia, tirará o Senhor o enfeite dos anéis dos tornozelos, e as toucas, e os ornamentos em forma de meia-lua; os pendentes, e os braceletes, e os véus esvoaçantes; os turbantes, as cadeiazinhas para os passos, as cintas, as caixinhas de perfumes e os amuletos; os sinetes e as joias pendentes do nariz; os vestidos de festa, os mantos, os xales e as bolsas; os espelhos, as camisas finíssimas, os atavios de cabeça e os véus grandes.

\textbf{KJ}: Moreover the LORD saith, Because the daughters of Zion are haughty, and walk with stretched forth necks and wanton eyes, walking and mincing as they go, and making a tinkling with their feet: Therefore the Lord will smite with a scab the crown of the head of the daughters of Zion, and the LORD will discover their secret parts. In that day the Lord will take away the bravery of their tinkling ornaments about their feet, and their cauls, and their round tires like the moon, the chains, and the bracelets, and the mufflers, the bonnets, and the ornaments of the legs, and the headbands, and the tablets, and the earrings, the rings, and nose jewels, the changeable suits of apparel, and the mantles, and the wimples, and the crisping pins, the glasses, and the fine linen, and the hoods, and the vails.

\subsection*{Is 5.25:} 
 \addcontentsline{toc}{subsection}{ }
 \begin{quote} 
 Por isso se acendeu a ira do Senhor contra o seu povo, e estendeu a sua mão contra ele, e o feriu, de modo que as montanhas tremeram, e os seus cadáveres se fizeram como lixo no meio das ruas\uwave{;} com tudo isto não tornou atrás a sua ira, mas a sua mão ainda está estendida.

RC: Pelo que se acendeu a ira do SENHOR contra o seu povo, e estendeu a mão contra ele e o feriu; e as montanhas tremeram, e os seus cadáveres eram como monturo no meio das ruas; com tudo isto não tornou atrás a sua ira, mas ainda está alçada a sua mão.

RA: Por isso, se acende a ira do SENHOR contra o seu povo, povo contra o qual estende a mão e o fere, de modo que tremem os montes e os seus cadáveres são como monturo no meio das ruas. Com tudo isto não se aplaca a sua ira, mas ainda está estendida a sua mão.

KJ: Therefore is the anger of the LORD kindled against his people, and he hath stretched forth his hand against them, and hath smitten them: and the hills did tremble, and their carcases were torn in the midst of the streets. For all this his anger is not turned away, but his hand is stretched out still.

\subsection*{Is 6.13:} 
 \addcontentsline{toc}{subsection}{ }
 \begin{quote} 
 Porém ainda a décima parte ficará nela, e tornará a ser pastada; e como o carvalho, e como a azinheira, que [,] depois de se desfolharem\uwave{,} ainda ficam firmes, assim a santa semente será a firmeza dela.

RA: Mas, se ainda ficar a décima parte dela, tornará a ser destruída. Como terebinto e como carvalho, dos quais, depois de derribados, ainda fica o toco, assim a santa semente é o seu toco.

RC: Mas, se ainda a décima parte dela ficar, tornará a ser pastada; como o carvalho e como a azinheira, que, depois de se desfolharem, ainda ficam firmes, assim a santa semente será a firmeza dela.

KJ: But yet in it shall be a tenth, and it shall return, and shall be eaten: as a teil tree, and as an oak, whose substance is in them, when they cast their leaves: so the holy seed shall be the substance thereof.

\subsection*{Is 7.9:} 
 \addcontentsline{toc}{subsection}{ }
 \begin{quote} 
 Entretanto a cabeça de Efraim será Samaria, e a cabeça de Samaria o filho de Remalias\uwave{;} se não o crerdes, certamente não haveis de permanecer.}

RC: Entretanto, a cabeça de Efraim será Samaria, e a cabeça de Samaria, o filho de Remalias; se o não crerdes, certamente, não ficareis firmes.

RA: Entretanto, a capital de Efraim será Samaria, e o cabeça de Samaria, o filho de Remalias; se o não crerdes, certamente, não permanecereis.

KJ: And the head of Ephraim is Samaria, and the head of Samaria is Remaliah’s son. If ye will not believe, surely ye shall not be established.

\subsection*{Is 7.23:} 
 \addcontentsline{toc}{subsection}{ }
 \begin{quote} 
 Sucederá também naquele dia que todo o lugar\uwave{,} em que houver mil vides, do valor de mil siclos de prata, será para as sarças e para os espinheiros.}

RC: Sucederá, também, naquele dia, que todo lugar em que houver mil vides do valor de mil moedas de prata será para sarças e para espinheiros.

RA: Também, naquele dia, todo lugar em que houver mil vides, do valor de mil siclos de prata, será para espinheiros e abrolhos.

KJ: And it shall come to pass in that day, that every place shall be, where there were a thousand vines at a thousand silverlings, it shall even be for briers and thorns.

\subsection*{Is 14.12:} 
 \addcontentsline{toc}{subsection}{ }
 \begin{quote} 
 Como caíste desde o céu, ó \uwave{estrela da manhã}, filha da alva! Como foste cortado por terra, tu que debilitavas as nações!}

Obrigado pelo detalhe. Lúcifer é muito melhor tradução do que ``Estrela
da manhã’’ que se aplica ao Senhor noutras passagens (Apoc 22:16). Essa
é a razão importante porque os tradutores da King James Version
(Inglesa) adotaram ``Lúcifer’’. Além disso, ``alva’’ consta do hebraico,
mas ``manhã’’ (em estrela da manhã) não consta, isto é, só aparece uma
vez e não duas o termo manhã (ou alva, que é o mesmo). Por isso, como
está na KJV é o correto (O Lucifer, son of the morning). Temos para
esse termo o voto de TODAS as versões da Reforma, o que é um voto de
muito peso! Agora, quando as versões modernas andam INVENTANDO termos
à toa para as suas ALDRABICES o escândalo tem que ser desmascarado!

O termo ``Calvário’’ também vem da Vulgata e ninguém reclama. Aliás, as
versões novas já ``retraduziram’’ ``Calvário’’ para ``Caveira’’! Ora, que
fazer com tantos hinos e tantos sermões de outros pregadores do
passado que prezavam com muito amor o termo ``Calvário’’? E quanto às
milhares de almas que foram tocadas por aquele nome tão solene?

O que interessa é saber se o Autor da Bíblia ACEITA um termo
latinizado ou do ``grego’’, ou do ``original’’. Existem outros nomes na
Bíblia que foram apenas transliterados e o Espírito Santo os aceitou,
nas versões que têm dado BOM FRUTO através dos séculos (Ap 3:8). O
que devemos rejeitar é que chamar a Satanás ``a estrela da manhã’’ ou da
``alva’’, não foi aprovado (nem poderia ser!) pelo Autor da Bíblia.

As versões modernas ATABALHOARAM todo o texto e não são documentos
inspirados para serem levados a sério pela igreja. Veja ``Inferno’’, que
foi ``DESTRADUZIDO’’ de volta para ``Hades’’ ou ``Sheol’’! Quem deu
autoridade às novas versões para ``DESTRADUZIR’’ aqueles termos
IMPORTANTÍSSIMOS das Escrituras?! É isso que desejo inquirir da NVI e
das outras!

Acontece que envio o meu material a vários colaboradores da NVI e
NENHUM se atreve a DEFENDER o seu próprio produto! Em vez de
defenderem o seu produto, voltam-se contra nós, dizendo que somos por
demais agressivos, rudes, e sem amor, etc. Ora, eles estão querendo
ROUBAR a nossa Bíblia do nosso coração e de tantos outros crentes,
substituindo-a pela sua colecção de ``bíblias’’ cheias de heresias!
Porque não REAGIR ofendidos e zangados?! Estão tentando ENVENENAR o
nosso coração e nós devemos continuar mansinhos?!!...

Um escândalo gravíssimo, onde se pode ver claramente a NEUTRALIDADE
SATÂNICA dos amigos das perversões modernas. Não sei se o irmão notou
nas versões modernas a omissão de ``não darás falso testemunho’’ em
Romanos 13:9. É o que as novas ``bíblias’’ fazem, atacando a
integridade do texto e do seu Autor.

Deus abençoe. Júlio. 


From: ``Josias Macedo Baraúna Júnior’’ <jbaraunajr@hotmail.com>

To: <enoque@mweb.co.za>
Sent: Tuesday, September 24, 2002 9:09 PM
Subject: Re: Rogo ajuda em ``história Bíblias depois Almeida’’
> Irmão Júlio,

> com respeito a Isaías 14.12, ``Lúcifer’’ é coisa da Vulgata Latina, ao que me parece (salvo melhor juízo!)

> Esta expressão é tradução de ``eosphoros’’, um termo grego da Septuaginta, que traduziu ``helel’’. Essa palavra provém da mesma raiz do verbo ``gemer’’em Zc 11.2.

> Isto não significa que eu não veja o inimigo de nossas almas neste texto. Muito pelo contrário. Somente acho que ``Lúcifer’’ seja um nome inventado para ele. ``Helel filho do Céu ou do Oriente’’ é, na minha opinião, a tradução literal de Is 14.12.

----- Original Message ----- 

From: ``Humberto Rafeiro’’ <hrafeiro@ilovejesus.net>

Caro irmão.

Aqui vai uma coisa que lhe vai agradar; é uma versão espanhola do ano
de 1596[SEV-Las 1596 Sagradas Escrituras]; que tirei da Biblia Online
que tenho em cd, e coloquei aqui os versículos que o ir. gostava de
saber [falta saber como era Almeida Original; mas visto que Almeida TB
se apoio em versões espanholas, holandesas; .. ; PESSOALMENTE, acho
que isso tb se reflectiria na versão original de Almeida; mas teremos
que arranjar provas mais conclusivas]:

Is 14:12 Cómo caíste del cielo, oh Lucero, hijo de la mañana! Cortado
fuiste por tierra, el que echabas suerte sobre los gentiles. [Eu
pensava que a primeira Bíblia a usar o termo ``Lúcifer’’ tinha sido a
KJV, mas me enganei, o termo é já anterior à KJV)

Muitíssimo grato pelo material enviado! Acontece que recentemente um
irmão me deu um CD com as bíblias da E-Sword onde tem duas em
Espanhol. Já chequei as duas e descobri que têm Lúcifer. Daí, podemos
concluir sem medo que a primeira Almeida tinha também Lúcifer. Só que
``Deus é Espírito’’ em vez de ``UM Espírito’’ não consta. Claro que
poderemos justificar as variantes aí, mas ``Lúcifer’’ preocupava-me,
visto que as moderninhas atacam o Senhor Jesus nesse verso. Mas,
continuemos a pesquisa.

Deus abençoe. Júlio.

\subsection*{Is 19.22-23:} 
 \addcontentsline{toc}{subsection}{ }
 \begin{quote} 
 E ferirá o Senhor ao Egito, ferirá e o curará; e converter-se-ão ao Senhor, e mover-se-á às suas orações, e os curará\uwave{;} $^{\mathrm{23}}$naquele dia haverá estrada do Egito até à Assíria, e os assírios virão ao Egito, e os egípcios irão à Assíria; e os egípcios servirão com os assírios.}

\subsection*{Is 21.8:} 
 \addcontentsline{toc}{subsection}{ }
 \begin{quote} 
 E clamou: \uwave{Um leão, meu Senhor!} Sobre a torre de vigia estou em pé continuamente de dia, e de guarda me ponho noites inteiras.}

RA: Então, o atalaia gritou como um leão: Senhor, sobre a torre de vigia estou em pé continuamente durante o dia e de guarda me ponho noites inteiras.

RC: 8 E clamou como um leão: Senhor, sobre a torre de vigia estou em pé continuamente de dia e de guarda me ponho noites inteiras.

KJ: 8 And he cried, A lion: My lord, I stand continually upon the watchtower in the daytime, and I am set in my ward whole nights:

\subsection*{Is 22.12:} 
 \addcontentsline{toc}{subsection}{ }
 \begin{quote} 
 E o Senhor Deus dos Exércitos\uwave{,} chamou naquele dia para chorar e para prantear, e para raspar a cabeça, e cingir com o cilício.}

KJ: And in that day did the Lord GOD of hosts call to weeping, and to mourning, and to baldness, and to girding with sackcloth:

\subsection*{Is 35.8:} 
 \addcontentsline{toc}{subsection}{ }
 \begin{quote} 
 E ali haverá uma estrada, um caminho, que se chamará o caminho santo; o imundo não passará por ele, \uwave{mas será para aqueles}; os caminhantes, até mesmo os loucos, não errarão.}

KJ: And an highway shall be there, and a way, and it shall be called The way of holiness; the unclean shall not pass over it; but it shall be for those: the wayfaring men, though fools, shall not err therein.

RA: E ali haverá bom caminho, caminho que se chamará o Caminho Santo; o imundo não passará por ele, pois será somente para o seu povo; quem quer que por ele caminhe não errará, nem mesmo o louco.

RC: E ali haverá um alto caminho, um caminho que se chamará O Caminho Santo; o imundo não passará por ele, mas será para o povo de Deus; os caminhantes, até mesmo os loucos, não errarão.

\subsection*{Is 36.12:} 
 \addcontentsline{toc}{subsection}{ }
 \begin{quote} 
 Rabsaqué, porém, disse: Porventura mandou-me o meu \uwave{Senhor} ao teu senhor e a ti, para dizer estas palavras e não antes aos homens que estão assentados sobre o muro, para que comam convosco o seu esterco, e bebam a sua urina?}
Minúsculo em todas as versões.
Ver King James.

\subsection*{Is 37.9:} 
 \addcontentsline{toc}{subsection}{ }
 \begin{quote} 
 E\uwave{,} ouviu ele dizer que Tiraca, rei da Etiópia, tinha saído para lhe fazer guerra. Assim que ouviu isto, enviou mensageiros a Ezequias, dizendo:\ldots}

RC: E ouviu dizer que Tiraca, rei da Etiópia, tinha saído para lhe fazer guerra. Assim que ouviu isso, enviou mensageiros a Ezequias, dizendo:

RA: O rei ouviu que, a respeito de Tiraca, rei da Etiópia, se dizia: Saiu para guerrear contra ti. Assim que ouviu isto, enviou mensageiros a Ezequias, dizendo:

KJ: And he heard say concerning Tirhakah king of Ethiopia, He is come forth to make war with thee. And when he heard it, he sent messengers to Hezekiah, saying,

\subsection*{Is 38.3:} 
 \addcontentsline{toc}{subsection}{ }
 \begin{quote} 
 E disse: Ah! Senhor, peço-te, lembra-te agora\uwave{,} de que andei diante de ti em verdade, e com coração perfeito, e fiz o que era reto aos teus olhos. E chorou Ezequias muitíssimo.}

RC: E disse: Ah! SENHOR, lembra-te, peço-te, de que andei diante de ti em verdade e com coração perfeito e fiz o que era reto aos teus olhos. E chorou Ezequias muitíssimo.

\subsection*{Is 38.14:} 
 \addcontentsline{toc}{subsection}{ }
 \begin{quote} 
 Como o grou, ou a andorinha, assim eu chilreava, e gemia como a pomba; alçava os meus olhos ao alto\uwave{;} ó Senhor, ando oprimido, fica por meu fiador.}

RA: Como a andorinha ou o grou, assim eu chilreava e gemia como a pomba; os meus olhos se cansavam de olhar para cima. Ó Senhor, ando oprimido, responde tu por mim.

RC: Como o grou ou a andorinha, assim eu chilreava e gemia como a pomba; alçava os olhos ao alto; ó Senhor, ando oprimido! Fica por meu fiador.

KJ: Like a crane or a swallow, so did I chatter: I did mourn as a dove: mine eyes fail with looking upward: O LORD, I am oppressed; undertake for me.

\subsection*{Is 40.20:} 
 \addcontentsline{toc}{subsection}{ }
 \begin{quote} 
 O empobrecido, que não pode oferecer tanto, escolhe madeira que não se apodrece; \uwave{artífice sábio busca, para gravar uma imagem que não se pode mover}.}

RA: O sacerdote idólatra escolhe madeira que não se corrompe e busca um artífice perito para assentar uma imagem esculpida que não oscile.

RC: O empobrecido, que não pode oferecer tanto, escolhe madeira que não se corrompe; artífice sábio busca, para gravar uma imagem que se não pode mover.

KJ: He that is so impoverished that he hath no oblation chooseth a tree that will not rot; he seeketh unto him a cunning workman to prepare a graven image, that shall not be moved.

\subsection*{Jr 1.9-10:} 
 \addcontentsline{toc}{subsection}{ }
 \begin{quote} 
 E estendeu o SENHOR a sua mão, e tocou-me na boca; e disse-me o SENHOR: Eis que ponho as minhas palavras na tua boca\uwave{;} $^{\mathrm{10}}$olha, ponho-te neste dia sobre as nações, e sobre os reinos, para arrancares, e para derrubares, e para destruíres, e para arruinares; e também para edificares e para plantares.}

Não seria ponto aí em vez de ponto-e-vírgula?

RC: E estendeu o SENHOR a mão, tocou-me na boca e disse-me o SENHOR: Eis que ponho as minhas palavras na tua boca. Olha, ponho-te neste dia sobre as nações e sobre os reinos, para arrancares, e para derribares, e para destruíres, e para arruinares; e também para edificares e para plantares.

RA: Depois, estendeu o SENHOR a mão, tocou-me na boca e o SENHOR me disse: Eis que ponho na tua boca as minhas palavras. Olha que hoje te constituo sobre as nações e sobre os reinos, para arrancares e derribares, para destruíres e arruinares e também para edificares e para plantares.

KJ: Then the LORD put forth his hand, and touched my mouth. And the LORD said unto me, Behold, I have put my words in thy mouth. See, I have this day set thee over the nations and over the kingdoms, to root out, and to pull down, and to destroy, and to throw down, to build, and to plant.

\subsection*{Jr 7.12:} 
 \addcontentsline{toc}{subsection}{ }
 \begin{quote} 
 Mas ide agora ao meu lugar, que estava em Siló, onde, \uwave{ao princípio}, fiz habitar o meu nome, e vede o que lhe fiz, por causa da maldade do meu povo Israel.}
Ao princípio ou no princípio?

KJ: But go ye now unto my place which was in Shiloh, where I set my name at the first, and see what I did to it for the wickedness of my people Israel.

RC: Mas ide agora ao meu lugar, que estava em Siló, onde, no princípio, fiz habitar o meu nome, e vede o que lhe fiz, por causa da maldade do meu povo de Israel.

RA: Mas ide agora ao meu lugar que estava em Siló, onde, no princípio, fiz habitar o meu nome, e vede o que lhe fiz, por causa da maldade do meu povo de Israel.

\subsection*{Jr 13.27:} 
 \addcontentsline{toc}{subsection}{ }
 \begin{quote} 
 Já vi as tuas abominações, e os teus adultérios, e os teus rinchos, e a enormidade da tua prostituição sobre os outeiros no campo\uwave{;} ai de ti, Jerusalém! Até quando ainda não te purificarás?}

Ponto em vez de ponto-e-vírgula.

KJ: I have seen thine adulteries, and thy neighings, the lewdness of thy whoredom, and thine abominations on the hills in the fields. Woe unto thee, O Jerusalem! wilt thou not be made clean? when shall it once be?

RA: Tenho visto as tuas abominações sobre os outeiros e no campo, a saber, os teus adultérios, os teus rinchos e a luxúria da tua prostituição. Ai de ti, Jerusalém! Até quando ainda não te purificarás?

RC: Vi as tuas abominações, e os teus adultérios, e os teus rinchos, e a enormidade da tua prostituição sobre os outeiros no campo; ai de ti, Jerusalém! Não te purificarás? Até quando ainda?

\subsection*{Jr 15.14-15:} 
 \addcontentsline{toc}{subsection}{ }
 \begin{quote} 
 E te farei passar aos teus inimigos numa terra que não conheces; porque o fogo se acendeu em minha ira, e sobre vós arderá; $^{\mathrm{15}}$tu, ó Senhor, o sabes; lembra-te de mim, e visita-me, e vinga-me dos meus perseguidores; não me arrebates por tua longanimidade; sabe que por amor de ti tenho sofrido afronta.}

Ponto parágrafo, depois de ``e sobre vós arderá;’’.

KJ: And I will make thee to pass with thine enemies into a land which thou knowest not: for a fire is kindled in mine anger, which shall burn upon you.

O LORD, thou knowest: remember me, and visit me, and revenge me of my persecutors; take me not away in thy longsuffering: know that for thy sake I have suffered rebuke.

RA: Levar-te-ei com os teus inimigos para a terra que não conheces; porque o fogo se acendeu em minha ira e sobre vós arderá.

Tu, ó SENHOR, o sabes; lembra-te de mim, ampara-me e vinga-me dos meus perseguidores; não me deixes ser arrebatado, por causa da tua longanimidade; sabe que por amor de ti tenho sofrido afrontas.

RC: E levarei a ti com os teus inimigos para a terra que não conheces; porque o fogo se acendeu em minha ira e sobre vós arderá.

Tu, ó SENHOR, o sabes; lembra-te de mim, e visita-me, e vinga-me dos meus perseguidores; não me arrebates, por tua longanimidade; sabe que, por amor de ti, tenho sofrido afronta.

\subsection*{Jr 18.3-4:} 
 \addcontentsline{toc}{subsection}{ }
 \begin{quote} 
 E desci à casa do oleiro, e eis que ele estava fazendo a sua obra sobre as rodas\uwave{,} 4como o vaso, que ele fazia de barro, quebrou-se na mão do oleiro, tornou a fazer dele outro vaso, conforme o que pareceu bem aos olhos do oleiro fazer.}

Ponto em vez de vírgula.

RC: E desci à casa do oleiro, e eis que ele estava fazendo a sua obra sobre as rodas. Como o vaso que ele fazia de barro se quebrou na mão do oleiro, tornou a fazer dele outro vaso, conforme o que pareceu bem aos seus olhos fazer.

KJ: Then I went down to the potter’s house, and, behold, he wrought a work on the wheels. And the vessel that he made of clay was marred in the hand of the potter: so he made it again another vessel, as seemed good to the potter to make it.

RA: Desci à casa do oleiro, e eis que ele estava entregue à sua obra sobre as rodas. Como o vaso que o oleiro fazia de barro se lhe estragou na mão, tornou a fazer dele outro vaso, segundo bem lhe pareceu.

\subsection*{Jr 20.1:} 
 \addcontentsline{toc}{subsection}{ }
 \begin{quote} 
 E PASUR, filho de Imer, o sacerdote, que havia sido nomeado \uwave{presidente} na casa do Senhor, ouviu a Jeremias, que profetizava estas palavras.}

Presidente ou Governador Chefe, conforme KJ. RA E RC também traduzem por ``Presidente’’.

KJ: Now Pashur the son of Immer the priest, who was also chief governor in the house of the LORD, heard that Jeremiah prophesied these things.

\subsection*{Jr 26.19:} 
 \addcontentsline{toc}{subsection}{ }
 \begin{quote} 
 Mataram-no, porventura, Ezequias, rei de Judá, e todo o Judá? Antes não temeu ao Senhor, e não implorou o favor do Senhor? E o Senhor não se arrependeu do mal que falara contra eles? Nós\uwave{,} fazemos um grande mal contra as nossas almas.}

Existe ou não esta vírgula?

KJ: Did Hezekiah king of Judah and all Judah put him at all to death? did he not fear the LORD, and besought the LORD, and the LORD repented him of the evil which he had pronounced against them? Thus might we procure great evil against our souls.

RA: Mataram-no, acaso, Ezequias, rei de Judá, e todo o Judá? Antes, não temeu este ao SENHOR, não implorou o favor do SENHOR? E o SENHOR não se arrependeu do mal que falara contra eles? E traríamos nós tão grande mal sobre a nossa alma?

RC: Mataram-no, porventura, Ezequias, rei de Judá, e todo o Judá? Antes, não temeu este ao SENHOR e não implorou o favor do SENHOR? E o SENHOR se arrependeu do mal que falara contra eles. E nós não fazemos um grande mal contra a nossa alma?

\subsection*{Jr 27.4:} 
 \addcontentsline{toc}{subsection}{ }
 \begin{quote} 
 E lhes ordenarás\uwave{,} que digam aos seus senhores: Assim diz o Senhor dos Exércitos, o Deus de Israel: Assim direis a vossos senhores:\ldots}
Não há essa vírgula.

RC: E lhes darás uma mensagem para seus senhores, dizendo: Assim diz o SENHOR dos Exércitos, o Deus de Israel: Assim direis a vossos senhores:

RA: Ordena-lhes que digam aos seus senhores: Assim diz o SENHOR dos Exércitos, o Deus de Israel: Assim direis a vossos senhores:

KJ: And command them to say unto their masters, Thus saith the LORD of hosts, the God of Israel; Thus shall ye say unto your masters;

\subsection*{Jr 27.8-9:} 
 \addcontentsline{toc}{subsection}{ }
 \begin{quote} 
 E acontecerá que, se alguma nação e reino não servirem o mesmo Nabucodonosor, rei de Babilônia, e não puserem o seu pescoço debaixo do jugo do rei de Babilônia, a essa nação castigarei com espada, e com fome, e com peste, diz o Senhor, até que a consuma pela sua mão\uwave{;} $^{\mathrm{9}}$e vós não deis ouvidos aos vossos profetas, e aos vossos adivinhos, e aos vossos sonhos, e aos vossos agoureiros, e aos vossos encantadores, que vos falam, dizendo: Não servireis ao rei de Babilônia.}

Ponto depois de aaté que a consuma pela sua mão;..

RC: E acontecerá que, se alguma nação e reino não servirem o mesmo Nabucodonosor, rei da Babilônia, e não puserem o pescoço debaixo do jugo do rei da Babilônia, visitarei com espada, e com fome, e com peste essa nação, diz o SENHOR, até que a consuma pelas suas mãos. E não deis ouvidos aos vossos profetas, e aos vossos adivinhos, e aos vossos sonhos, e aos vossos agoureiros, e aos vossos encantadores, que vos falam, dizendo: Não servireis ao rei da Babilônia.

KJ: And it shall come to pass, that the nation and kingdom which will not serve the same Nebuchadnezzar the king of Babylon, and that will not put their neck under the yoke of the king of Babylon, that nation will I punish, saith the LORD, with the sword, and with the famine, and with the pestilence, until I have consumed them by his hand. Therefore hearken not ye to your prophets, nor to your diviners, nor to your dreamers, nor to your enchanters, nor to your sorcerers, which speak unto you, saying, Ye shall not serve the king of Babylon:

\subsection*{Jr 29.23:} 
 \addcontentsline{toc}{subsection}{ }
 \begin{quote} 
 \ldots porquanto fizeram loucura em Israel, e cometeram adultério com as mulheres dos seus vizinhos, e anunciaram falsamente\uwave{,} em meu nome uma palavra\uwave{,} que não lhes mandei, e eu o sei e sou testemunha disso, diz o Senhor.}

Não há essas vírgulas.

RC: Porquanto fizeram loucura em Israel, e cometeram adultério com as mulheres de seus companheiros, e anunciaram falsamente em meu nome palavras que não lhes mandei dizer; e eu o sei e sou testemunha disso, diz o SENHOR.

KJ: Because they have committed villany in Israel, and have committed adultery with their neighbours’ wives, and have spoken lying words in my name, which I have not commanded them; even I know, and am a witness, saith the LORD.

\subsection*{Jr 47.6:} 
 \addcontentsline{toc}{subsection}{ }
 \begin{quote} 
 Ah\uwave{;} espada do Senhor! Até quando deixarás de repousar? Volta para a tua bainha, descansa, e aquieta-te.}

De acordo com a RC e RA: Ah! Espada do Senhor!

KJ: O thou sword of the LORD, how long will it be ere thou be quiet? put up thyself into thy scabbard, rest, and be still.

\subsection*{Jr 48.3:} 
 \addcontentsline{toc}{subsection}{ }
 \begin{quote} 
 Voz de clamor de Horonaim\uwave{;} ruína e grande destruição!}

Não é ponto-e-vírgula, mas dois pontos ou vírgula.

KJ: A voice of crying shall be from Horonaim, spoiling and great destruction.

RA: Há gritos de Horonaim: Ruína e grande destruição!

RC: Voz de grito de Horonaim: Ruína e grande destruição!

\subsection*{Jr 50.25:} 
 \addcontentsline{toc}{subsection}{ }
 \begin{quote} 
 O Senhor abriu o seu depósito, e tirou os instrumentos da sua indignação; porque o Senhor Deus dos Exércitos\uwave{,} tem uma obra a realizar na terra dos caldeus.}

Não há vírgula depois de ``Exércitos’’.

RC: O SENHOR abriu o seu tesouro e tirou os instrumentos da sua indignação, porque o Senhor JEOVÁ dos Exércitos, tem uma obra a realizar na terra dos caldeus.

KJ: The LORD hath opened his armoury, and hath brought forth the weapons of his indignation: for this is the work of the Lord GOD of hosts in the land of the Chaldeans.

RA: O SENHOR abriu o seu arsenal e tirou dele as armas da sua indignação; porque o Senhor, o SENHOR dos Exércitos, tem obra a realizar na terra dos caldeus.

\subsection*{Jr 50.27:} 
 \addcontentsline{toc}{subsection}{ }
 \begin{quote} 
 Matai a todos os seus novilhos, desçam \uwave{a} matança. Ai deles, porque veio o seu dia, o tempo do seu \uwave{castigo}!}

Falta a crase: ``desçam à matança’’.

RC: Matai à espada a todos os seus novilhos, que eles desçam ao degoladouro; ai deles! Porque veio o seu dia, o tempo da sua visitação.

RA: Matai à espada a todos os seus touros, aos seus valentes; desçam eles para o matadouro; ai deles! Pois é chegado o seu dia, o tempo do seu castigo.

KJ: Slay all her bullocks; let them go down to the slaughter: woe unto them! for their day is come, the time of their visitation.

Obs.: Ainda neste versículo, a KJ registra ``visitação’’ em vez de ``castigo’’ da ACFiel.

\subsection*{Jr 51.16-17:} 
 \addcontentsline{toc}{subsection}{ }
 \begin{quote} 
 Fazendo ele ouvir a sua voz, grande estrondo de águas há nos céus, e faz subir os vapores desde o fim da terra; faz os relâmpagos com a chuva, e tira o vento dos seus tesouros\uwave{,} $^{\mathrm{7}}$embrutecido é todo o homem, no seu conhecimento; envergonha-se todo o artífice da imagem de escultura; porque a sua imagem de fundição é mentira, e nelas não há espírito.}

Entre um versículo e outro há um ponto.

RA: Fazendo ele ribombar o trovão, logo há tumulto de águas no céu, e sobem os vapores das extremidades da terra; ele cria os relâmpagos para a chuva e dos seus depósitos faz sair o vento. Todo homem se tornou estúpido e não tem saber; todo ourives é envergonhado pela imagem que esculpiu; pois as suas imagens são mentira, e nelas não há fôlego.

RC: Fazendo ele ouvir a sua voz, grande estrondo de águas há nos céus, e sobem os vapores desde o fim da terra; faz os relâmpagos com a chuva e tira o vento dos seus tesouros. Embruteceu-se todo homem e não tem ciência; envergonhou-se todo ourives de imagem de escultura, porque a sua imagem de fundição é mentira, e não há espírito em nenhuma delas.

KJ: When he uttereth his voice, there is a multitude of waters in the heavens; and he causeth the vapours to ascend from the ends of the earth: he maketh lightnings with rain, and bringeth forth the wind out of his treasures. Every man is brutish by his knowledge; every founder is confounded by the graven image: for his molten image is falsehood, and there is no breath in them.

\subsection*{Jr 52.4:} 
 \addcontentsline{toc}{subsection}{ }
 \begin{quote} 
 E aconteceu\uwave{,} que no ano nono do seu reinado, no décimo mês, no décimo dia do mês, veio Nabucodonosor, rei de Babilônia, contra Jerusalém, ele e todo o seu exército, e se acamparam contra ela, e levantaram contra ela trincheiras ao redor.}

Não há essa vírgula.

RA: Sucedeu que, em o nono ano do reinado de Zedequias, aos dez dias do décimo mês, Nabucodonosor, rei da Babilônia, veio contra Jerusalém, ele e todo o seu exército, e se acamparam contra ela, e levantaram contra ela tranqueiras em redor.

RC: E aconteceu, no ano nono do seu reinado, no mês décimo, no décimo dia do mês, que veio Nabucodonosor, rei da Babilônia, contra Jerusalém, ele e todo o seu exército, e se acamparam contra ela e levantaram contra ela tranqueiras ao redor.

KJ: And it came to pass in the ninth year of his reign, in the tenth month, in the tenth day of the month, that Nebuchadrezzar king of Babylon came, he and all his army, against Jerusalem, and pitched against it, and built forts against it round about.

\subsection*{Lm 1.9,11:} 
 \addcontentsline{toc}{subsection}{ }
 \begin{quote} 
 A sua imundícia está nas suas saias; nunca se lembrou do seu fim; por isso foi pasmosamente abatida, não tem consolador\uwave{;} vê, Senhor , a minha aflição, porque o inimigo se tem engrandecido. $^{\mathrm{11}}$Todo o seu povo anda suspirando, buscando o pão; deram as suas coisas mais preciosas a troco de mantimento para restaurarem a alma\uwave{;} vê, Senhor , e contempla, que sou desprezível.}

\subsection*{Dn 1.4:} 
 \addcontentsline{toc}{subsection}{ }
 \begin{quote} 
 E os caldeus disseram ao rei em aramáico: Ó rei, vive eternamente! Dize o sonho a teus servos, e daremos a interpretação.}

``Aramaico’’, sem acento!

\subsection*{Dn 4.18:} 
 \addcontentsline{toc}{subsection}{ }
 \begin{quote} 
 Este sonho eu, rei Nabucodonosor[,] vi. Tu, pois, Beltessazar, dize a interpretação, porque todos os sábios do meu reino não puderam fazer-me saber a sua interpretação, mas tu podes; pois há em ti o espírito dos deuses santos.}
Falta uma vírgula depois de Nabucodonosor.

\subsection*{Dn 6.7:} 
 \addcontentsline{toc}{subsection}{ }
 \begin{quote} 
 Todos os \uwave{presidentes} do reino, os capitães e príncipes, conselheiros e governadores, concordaram em promulgar um edito real e confirmar a proibição que qualquer que, por espaço de trinta dias, fizer uma petição a qualquer deus, ou a qualquer homem, e não a ti, ó rei, seja lançado na cova dos leões.}

\subsection*{Zc 9.9:} 
 \addcontentsline{toc}{subsection}{ }

Alegra-te muito, ó filha de Sião; exulta, ó filha de Jerusalém; eis que o teu rei virá a ti, justo e SALVO \{3467 Yâsha\}, pobre, e montado sobre um jumento, e sobre um jumentinho, filho de jumenta.

Alegra-te muito, ó filha de Sião; exulta, ó filha de Jerusalém; eis que o teu rei virá a ti, justo e TENDO SALVAÇÃO \{3467 Yâsha\}, pobre, e montado sobre um jumento, e sobre um jumentinho, filho de jumenta.

Mesmo se o hebraico der lugar a ambas as traduções ``salvo’’ e
``salvador’’, toda a Bíblia aponta a segunda como melhor. Como nas KJV
(``having salvation’’) e na ARC 1948 (``salvador’’).


\vspace{24pt}

Queridos irmãos, são apenas estas as questões que coloco. Fiz tantas
recomendações de pontuação que não é pouco provável enganos e exageros
de minha parte. Às vezes eu mesmo me considero exigente e meticuloso
demais. Mas espero que a Sociedade Bíblica Trinitariana do Brasil
possa checar todas essas considerações que faço, bem como os demais
irmãos, e retenha o que for pertinente. Uma certeza tenho: urge há
muito uma revisão do nosso texto bíblico. Espero estar contribuindo
para isso.

\vspace{12pt}

Saudações a todos. Que a paz de nosso Senhor Jesus Cristo repouse
sobre todos vós.

\vspace{24pt}
