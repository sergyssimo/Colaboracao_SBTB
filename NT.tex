\chapter*{Introdução}
\addcontentsline{toc}{chapter}{Introdução}

Prezados irmãos em Cristo, graça e paz a todos!

Enfim, coloco à disposição da Sociedade Bíblica Trinitariana do
Brasil, e dos demais irmãos e colaboradores de nossa tão estimada
Bíblia Almeida Corrigida e Fiel, as minhas considerações e sugestões
ao Novo Testamento publicado por essa querida Sociedade. Basicamente
são críticas relacionadas à Gramática da Língua Portuguesa: termos
ultrapassados, erros ortográficos e/ou de pontuação, passagens
truncadas e que necesitam de uma melhor redação, etc.

Desde já, deixo claro que não sou professor de Português. Sou
taquígrafo do Tribunal de Contas do Estado do Rio de Janeiro --- lido
com textos, portanto. Todavia, as questões aqui abordadas que dizem
respeito à Gramática foram por mim dirimidas com revisores e
tradutores, também funcionários do Tribunal de Contas. Em especial,
agradeço ao escritor, tradutor e professor de Português, hoje meu
colega de trabalho, Ronaldo Redó Lanzillotti, que me ajudou em
diversos tópicos que ora compartilho com os irmãos.

Tomei a liberdade de fazer, em várias oportunidades, comparações com
outras versões da Bíblia em português, a saber: Revista e Atualizada
(RA), publicada pela Sociedade Bíblica do Brasil e a Bíblia de
Jerusalém (BJ), publicada pelas Edições Paulinas --- ambas, como todos
sabem, quero crer, traduções baseadas no Texto Crítico. E já explico
(antes que muitos se escandalizem e desistam de ler desde já): de
forma alguma considero o Texto Crítico superior ao Recebido, mas
pretendo demonstrar que, em várias ocasiões, por incrível que possa
parecer aos irmãos, a redação ou a tradução de uma determinada
passagem ou versículo no TC (representado aqui por essas duas versões)
se encontra melhor e até mesmo mais fiel ao grego do Texto Recebido do
que na ACFiel. Outrossim, escolhi essas duas por serem bem
distintas entre si no que diz respeito ao estilo literário.
Finalmente, não podia deixar de incluir a nossa tão estimada Bíblia King James~(AV) nessas minhas breves avaliações.

Confiram, portanto, vocês mesmos esses 66 itens que ponho à disposição
e consideração dos amados irmãos, ao mesmo tempo esperando um retorno
dessa crítica que faço e não me eximindo de estar errado em vários
pontos. Proponho que examinemos tudo e retenhamos aquilo que é bom
(1Ts 5.19-21).

\chapter{Evangelhos}
\section{Mateus}
\subsection*{Mt 5.26:}\label{ceitil}
\addcontentsline{toc}{subsection}{Mt 5.26}
\begin{quote}
 \small
 Em verdade te digo que de maneira nenhuma sairás dali enquanto não pagares o último \uwave{ceitil}
\end{quote}

Confesso que até há pouco tempo achava que ceitil era um tipo de moeda
da época, o que de forma alguma é o caso. Como veremos, mais dois
versículos traduzem por ceitil o nome de duas moedas utilizadas na
época de Jesus: o \emph{lepton} (do grego \emph{leptos}, ``pequeno,
fino''), a única moeda judaica mencionada no Novo Testamento e o
\emph{assarion} (ou, como é traduzido para o português, asse) romano.

Ora, o ceitil era uma antiga moeda portuguesa, que valia um sexto do
real português de então. Não faz nenhum sentido usar esse termo. A
tradução deve ser literal nestes casos e não associativa: havia
uma moeda, judaica, de bronze, que se se chamava lepto e que
correspondia a $1/8$ do \emph{assarion} (moeda de cobre romana) e fim
de papo.

\subsection*{Mt 5.31:}
\addcontentsline{toc}{subsection}{Mt 5.31}
\begin{quote}
 \small
 Também foi dito: Qualquer que deixar sua mulher, dê-lhe carta de \uwave{desquite}.
\end{quote}

A palavra desquite está completamente desatualizada do contexto atual. Não se usa mais tal termo. O mais correto aqui seria \texttt{divórcio}. A King James corrobora essa opinião: ``writing of divorcement''.

\subsection*{Mt 6.13:}
\addcontentsline{toc}{subsection}{Mt 6.13}
\begin{quote}
 \small
 E não nos \uwave{induzas} à tentação; mas livra-nos do mal\ldots
\end{quote}

O irmão Hélio de Menezes já havia alertado sobre essa tradução,
sugerindo ``conduzas'' em vez de ``induzas''. Não sei se foi acolhida
a sua sugestão pela Sociedade Bíblica Trinitariana do Brasil.

King James: ``And \emph{lead us} not into temptation, but deliver us
from evil''.

\subsection*{Mt 10.29:}
\begin{quote}
 \small
 Não se vendem dois passarinhos por um \uwave{ceitil}? e nenhum deles cairá em terra sem a vontade de vosso Pai.
\end{quote}

Idem \ref{ceitil}. Só que, neste caso, o grego registra outra moeda: o \emph{assarion} romano, ou, em nosso português, asse. Olha o TC dando um banho na gente aqui: eles traduzem fielmente por asse ou asses\label{asse} (Lc 12.6). Vamos reter o que é bom!

\subsection*{Mt 11.10:}
\textbf{Porque é este de quem está escrito: Eis que diante da tua face envio o meu \uwave{anjo}, que preparará diante de ti o teu  caminho.}
Eu sei que anjo também quer dizer mensageiro. Mas, neste versículo, faço apenas uma indagação aos tradutores sobre o porquê (sou muito curioso) da opção por ``anjo'' em detrimento de ``mensageiro'' (RA, BJ). A King James usa a palavra ``messenger''.

\subsection*{Mt 12.42:} 
\begin{quote} \small A \uwave{rainha do meio-dia} se levantará no dia do juízo com esta geração, e a condenará; porque veio dos confins da terra para ouvir a sabedoria de Salomão. E eis que está aqui quem é maior do que Salomão. \end{quote}

Neste versículo, creio que a tradução está errada e o correto seria:
RAINHA DO SUL. Na King James, temos: ``The queen of the south''. É interessante observar que a edição Revista e Corrigida da Imprensa Bíblica Brasileira (1987) também registra ``rainha do meio-dia'', mas se os irmãos forem consultar a Bíblia Online, software produzido pela Sociedade Bíblica do Brasil, na versão da Revista e Corrigida que lá aparece também consta ``rainha do sul''. Em grego, temos \emph{balissa tou notou}; e \emph{notos} quer dizer tanto região do sul como vento sul.

\subsection*{Mt 15.30-31:}
\begin{quote} \small E veio ter com ele grandes multidões, que traziam coxos, cegos, mudos, aleijados, e outros muitos, e os puseram aos pés de Jesus, e ele os sarou, de tal sorte\uwave{,} que a multidão se maravilhou vendo os mudos a falar, os aleijados sãos, os coxos a andar, e os cegos a ver; e glorificava o Deus de Israel. \end{quote}

Não há essa vírgula seguinte à locução ``de tal sorte''.

\subsection*{Mt 15.34:}
\begin{quote}
 \small
 E Jesus disse-lhes: Quantos pães tendes? E eles disseram: Sete, e uns poucos \uwave{de} peixinhos.
\end{quote}

O que essa preposição ``de'' está fazendo aí? Simplesmente: ``e uns poucos peixinhos'' ou ``e alguns peixinhos''.

\subsection*{Mt 18.28:} \label{denario}
\textbf{Saindo, porém, aquele servo, encontrou um dos seus
conservos, que lhe devia cem \uwave{dinheiros}, e, lançando mão dele, sufocava-o, dizendo: Paga-me o que me deves.}

Bem, essa é uma questão que gostaria de discutir com os amados irmãos:
denário ou dinheiro? Até porque o termo \emph{dinheiro}, usado como
unidade monetária se repete em vários outros versículos, como veremos
mais adiante. Na Revista e Corrigida, há uma nota de rodapé:
``denários''.

Não sei quanto aos irmãos, mas sempre ``doeu'' em meus ouvidos a
leitura deste e demais versículos semelhantes. Simplesmente não faz
nenhum sentido, pelo menos ao falante de língua portuguesa, uma frase
como \emph{cem dinheiros}. Ao leitor da King James, talvez sim, quando
lê: ``An hundred pence'', já que \emph{penny} é moeda inglesa e
\emph{pence} indica valor ou custo. Se bem que teria sido preferível
--- e não entendo por que não o foi --- que eles tivessem traduzido
\emph{denarion} por \texttt{denários} em vez de ``pence''.

Embora a palavra dinheiro tenha sua etmologia no lat.vulg. dinarius, o
denário era um tipo de moeda da época. Na época de Cristo, o dinheiro
era cunhado em três metais próprios: ouro, prata e cobre, bronze ou
latão. E havia moedas judaicas, gregas e romanas. Por exemplo, a
dracma (Lc 15.8) é uma moeda grega; o \emph{lepton}, moeda judaica e o
denário de prata era a moeda romana básica. O termo ``denário''
(\emph{deni} = dez por vez) deriva do fato de que a princípio era
equivalente em prata a dez asses de cobre.

Cito o Novo Dicionário da Bíblia: ``O bronze (em grego \emph{chalkos})
é a palavra usada em geral para significar dinheiro, como em Mc~6.8 e
12.41, mas como apenas as moedas de menor valor, como o `as' romano
(gr. \emph{assarion}) e o \emph{lepton} judaico eram cunhadas em
bronze, o termo mais comum e geral usado com o sentido de dinheiro no
Novo Testamento era `prata' (em grego, \emph{argyrion} --- Lc~9.3;
At~8.20, etc.)''.

Consultem, por exemplo, a passagem bem conhecida de 1Tm~6.10, que diz que o amor ao dinheiro é a raiz de todos os males. A frase ``amor ao dinheiro'' corresponde, no grego, ao termo \emph{philargyria}, ou ``amor, ou apego, afinidade, à prata''.

Ainda outros termos empregados como ``dinheiro'' no Novo Testamento:
\begin{itemize}
\item \emph{chr\~ema} -- propriedade, riqueza, mas também dinheiro (At~4.37; 8.18,20; 24.26).
\item \emph{kerma} -- ``troco pequeno'' (Jo 2.15).
\item \emph{nomisma} -- ``dinheiro introduzido no uso comum pela lei
 (\emph{nomos})'' Ex.: Mt 22.19: ``Mostrai-me a moeda do tributo. E
 eles lhe apresentaram \emph{um dinheiro}''. O ``dinheiro'', aqui,
 significa a moeda legal para o pagamento do imposto. Neste caso
 específico, sugeriria uma nota de rodapé explicativa.
\end{itemize}

Portanto, irmãos --- e não sei se me fiz muito prolixo --- da
Sociedade Bíblica Trinitariana do Brasil, não traduzam \emph{denarion}
por dinheiro, mas por denário mesmo, conforme as outras versões o têm
feito. Embora dinheiro possa significar a moeda corrente, segundo o
dicionário, não há sentido em se dizer ``cem dinheiros'', porque há um
numeral que o especifica, e não existe uma moeda chamada ``dinheiro; por outro lado, ``cem denários'' faz todo o sentido: o \texttt{denário} era a principal moeda romana na época de Jesus.


\subsection*{Mt 20.2:}
\begin{quote} \small E, ajustando com os trabalhadores a um \uwave{dinheiro} por dia, man\-dou-os para a sua vinha.\end{quote}
Idem \ref{denario}.


\subsection*{Mt 20.9,10,13:}
\begin{quote}
 \small
 E, chegando os que tinham ido perto da hora undécima, receberam um \uwave{dinheiro} cada um. Vindo, porém, os primeiros, cuidaram que haviam de receber mais; mas do mesmo modo receberam um \uwave{dinheiro} cada um. Mas ele, respondendo, disse a um deles: Amigo, não te faço agravo; não ajustaste tu comigo um \uwave{dinheiro}?
\end{quote}

Idem \ref{denario}.

\subsection*{Mt 23.7:}
\begin{quote}
 \small
 E as saudações nas praças, e o serem chamados pelos homens\uwave{;} Rabi, Rabi.
\end{quote}

Depois de ``homens'' não é ponto-e-vírgula, mas dois pontos.


\subsection*{Mt 24.15:}
\begin{quote}
 \small
Quando, pois, virdes que a abominação da desolação, de
que falou o profeta Daniel, está no lugar santo; quem lê, \uwave{atenda};\ldots
\end{quote}
``Quem lê, ENTENDA''. King James: ``understand''.

 Novamente, na Revista e Corrigida da Imprensa Bíblica Brasileira,
 1987, registra ``atenda''; mas na veiculada pela Sociedade Bíblica do
 Brasil na Bíblia Online, diz: ``quem lê, quem entenda''.

\subsection*{Mt 27.11:}\label{govern}
\begin{quote}
 \small
 E foi Jesus apresentado ao \uwave{presidente}, e o \uwave{presidente} o interrogou, dizendo: És tu o Rei dos Judeus? E disse-lhe Jesus: Tu o dizes.
\end{quote}

Não seria mais ortodoxo o uso de ``governador''? Excetuando-se a ACFiel, as demais versões usam ``governador''. King James, por exemplo: \emph{governor}. E na Revista e Corrigida, há uma nota de rodapé: ``Ou governador''. O termo grego aqui é \emph{hegemon}: ``guia, condutor; chefe, comandante; em Roma: imperador. Novo
Testamento: governador de província'' (Dicionário Grego-Português/Português-Grego, de Isidro Pereira). E vários versículos há ainda que empregam ``presidente'' em lugar de ``governador'' --- devo citá-los adiante.


\subsection*{Mt 27.54:}
\begin{quote}
 \small
 E o centurião e os que com ele guardavam a Jesus, vendo o terremoto, e as coisas que haviam sucedido, tiveram grande temor, e disseram: Verdadeiramente este \uwave{era FILHO DE DEUS}.
\end{quote}

Com a palavra, aqui, o irmão Hélio de Menezes, num de seus e-mails afortunadamente por mim recebidos:

``\ldots Verdadeiramente este era O (artigo definido masculino e singular)
FILHO DE DEUS''. O Texto Recebido diz: `alhywv 230 (ADV yeou 2316 (N-GSM)
uiov 5207 (N-NSM) hn 2258 5713 (V-IXI-3S) outov 3778 (D-NSM)'. A
construção e o contexto exigem o artigo definido. A tradução literal tem
que ser `em-a-verdade, (o) Filho de Deus era este.' Cristo não é `um
filho de Deus' (a NVI está definitivamente errada), não é `um dos filhos
de Deus' (os crentes e os anjos também o são), não é `um filho de um
deus' (deus, com inicial minúscula) (NASV 1963): Cristo é `\textbf{o}
Filho de Deus', é o total+exclusivo Filho de o total $+$ exclusivo Deus.

O TC diz: `alhywv uios yeou hn outov.' Mas, mesmo assim, também a
construção e o contexto recomendam o arquivo definido.''

\section{Marcos}\label{seteda}
\subsection*{Mc 4.38:}
\begin{quote}
 \small
E ele estava na popa, dormindo sobre uma almofada, e despertaram-no, dizendo-lhe: Mestre, \uwave{não se te dá} que pereçamos?
\end{quote}

Não é que esteja errada, mas vocês não acham essa construção -- ``não
se te dá'' -- um tanto arcaica e sem sentido nos dias de hoje (cf.
item 21). Não seria preferível: ``Mestre, não te importa que
pereçamos?'' ou outra semelhante. A King James registra: ``Master,
carest thou not that we perish?''.


\subsection*{Mc 5.25-27:}
\begin{quote}
    \small
E certa mulher que, havia doze anos,  tinha um fluxo de sangue, e que havia padecido muito  com muitos médicos, e despendido tudo quanto tinha, nada lhe  aproveitando isso, antes indo a pior; ouvindo falar  de Jesus, veio por detrás, entre a multidão, e tocou na sua veste.
\end{quote}

Ao trecho acima falta certa coesão textual. O versículo 26 é uma
intercalação. Não ficaria melhor redigido assim, sem alterar
praticamente nada: ``E certa mulher que, havia doze
anos, tinha um fluxo de sangue --- e que havia padecido
muito com muitos médicos, e despendido tudo quanto tinha, nada lhe
aproveitando isso, antes indo a pior --- ouvindo falar
de Jesus, veio por detrás, entre a multidão, e tocou na sua veste.'' Também poder-se-ia colocar o versículo intercalado entre parênteses. O que vocês acham?


\subsection*{Mc 6.37:}
\begin{quote}
    \small
Ele, porém, respondendo, lhes disse: Dai-lhes vós de
 comer. E eles disseram-lhe: Iremos nós, e compraremos
duzentos \uwave{dinheiros} de pão para lhes darmos de comer?
\end{quote}
Ver \ref{denario}.


\subsection*{Mc 9.3:}
\begin{quote}
    \small
E as suas vestes tornaram-se resplandecentes, extremamente brancas como a neve, tais como nenhum lavadeiro sobre a terra \uwave{os} poderia branquear.
\end{quote}

``\ldots \emph{as} poderia branquear''. Branquear AS roupas, não?


\subsection*{Mc 11.19:}\label{saiu}
\begin{quote}
    \small
E, sendo já tarde, \uwave{saiu para fora} da cidade.
\end{quote}

Esse talvez seja o erro de português mais comum em todas as versões das bíblias em português. Na versão editada pela Sociedade Bíblica Trinitariana do Brasil, conhecida como Almeida Corrigida e Fiel, aparece dezenas de vezes, tanto no Antigo como no Novo Testamentos.

Não sei quantos de vocês já se depararam com esta situação por ocasião
do estudo da Língua Portuguesa, mas ``sair para fora'' e ``entrar para
dentro'', bem como ``subir para cima'' e ``descer para baixo'' é
chamado de pleonasmo, ou mais: \emph{pleonasmo vicioso} de tão flagrante que
é. Não preciso explicar o porquê aos irmãos, pois todos sabem que sair
só pode ser para fora, e entrar só pode ser para dentro; o mesmo vale
para subir e descer --- só pode ser para cima ou para baixo,
respectivamente.

Confesso aos irmãos profundo constrangimento de ler, por exemplo,
João 11.43 em público, em voz alta mesmo, em nossa estimada versão:
``Lázaro, sai para fora.'' Que coisa horrível, meus irmãos! Vamos ter
um pouco mais de zelo por nossa língua!

A Revista e Atualizada, por exemplo, e que nós tanto criticamos,
redige corretamente: ``saíram da cidade''. Simplesmente. Já a Revista
e Corrigida, comete a mesma gafe. E, para quem entende um pouco de
inglês, a King James registra: ``he went out of the city'' --- embora
sejam dois os verbetes, ``went out'' significa simplesmente ``sair'',
e não ``sair para fora''.

\subsection*{Mc 12.14:}
\begin{quote}
    \small
E, chegando eles, disseram-lhe:
 Mestre, sabemos que és homem de verdade, \uwave{e de ninguém se te
 dá}, porque não olhas à aparência dos homens, antes com verdade
 ensinas o caminho de Deus; é lícito dar o tributo a César, ou não?
 Daremos, ou não daremos?
\end{quote}

O mesmo comentário que fiz no item \ref{seteda}  também se aplica aqui:
construção bem arcaica. Não sei se os irmãos concordam se não seria
possível dar uma ``lapidada'' neste trecho. Apenas para esclarecer a
frase, coloco a versão da Bíblia de Jerusalém (e, mais uma vez, não
estou sugerindo que se copie, até porque não consultei essa passagem
no grego para ver o que melhor ficaria no português): ``e não dás
preferência a ninguém''. Interessante é que a Revista e Atualizada também coloca ``e de ninguém se te dá''.

A King James registra: ``And when they were come, they say unto him,
Master, we know that thou art true, and carest for no man: for thou
regardest not the person of men, but teachest the way of God in truth:
Is it lawful to give tribute to Caesar, or not?''.

\subsection*{Mc 12.42:}
\begin{quote}
    \small
Vindo, porém, uma pobre viúva, deitou \uwave{duas pequenas
moedas}, que valiam meio \uwave{centavo}.
\end{quote}

Aqui vale um pouco o que anteriormente sobre se traduzir ``dinheiro''
em lugar de ``denário''. A palavra grega é ``quadrante'': uma moeda
romana equivalente a $1/4$ da moeda de cobre \emph{as} --- era a menor
das moedas romanas. As versões RA, BJ e RC registram ``quadrante'',
sendo que nesta última há uma nota: ``um centavo em moeda
brasileira''. Além disso, a tradução literal de ``duas pequenas
moedas'' é ``dois \emph{lepta}'' ou leptos.

Portanto, acho preferível a tradução como \texttt{quadrante} e \texttt{leptos} pelos mesmo motivos já expostos (itens \ref{ceitil} e \ref{asse}).

\subsection*{Mc 14.5:}
\begin{quote}
    \small
Porque podia vender-se por mais de trezentos \uwave{dinheiros}, e dá-lo aos pobres. E bramavam contra ela.
\end{quote}

Trezentos denários (ver \ref{denario}).

\subsection*{Mc 14.68:}
\begin{quote}
    \small
Mas ele negou-o, dizendo: Não o conheço, nem sei o que
dizes. E \uwave{saiu fora} ao alpendre, e o galo cantou.
    \end{quote}
    
Olha o ``sair fora'' ou ``para fora'' aí de novo. Que tal, ``e foi para fora''? Não ficaria melhor? (ver item \ref{saiu}).

\section{Lucas}
\subsection*{Lc 6.1:}
\begin{quote}
    \small
E aconteceu que, no \uwave{sábado segundo-primeiro},
passou pelas searas, e os seus discípulos iam arrancando espigas e,
esfregando-as com as mãos, as comiam.
\end{quote}

Irmãos, aqui faço apenas uma pergunta: o que é o ``sá\-ba\-do-se\-gun\-do-pri\-mei\-ro''? O Texto Crítico --- e longe de mim querer segui-lo, faço sempre a questão de frisar este ponto --- somente menciona ``num dia de sábado''. E se a ACFiel acrescenta algo, é porque de fato existe mais coisa que o TC fez questão de omitir. Mas o que seria, então, o ``sábado-segundo-primeiro''? Vamos ler em inglês
(KJ)? ``And it came to pass on the second sabbath after the first, that he went through the corn fields \ldots''. No meu parco conhecimento da língua inglesa, parece dizer ``no segundo sábado depois do primeiro''. O que os irmãos acham? Será que a melhor tradução é mesmo a do tal ``sábado-segundo-primeiro''?

\subsection*{Lc 10.35:}
\begin{quote}
    \small
E, partindo no outro dia, tirou dois \uwave{dinheiros}, e deu-os ao hospedeiro, e disse-lhe: Cuida dele; e tudo o que de mais gastares eu to pagarei quando voltar.
Ver item \ref{denario}.
    \end{quote}

\subsection*{Lc 12.6a:}
\begin{quote}
    \small
Não se vendem cinco passarinhos por dois \uwave{ceitis}?
\end{quote}

Dois ASSES. Ver itens \ref{ceitil} e \ref{asse}.

\subsection*{Lc 12.59:}
\begin{quote}
    \small
Digo-te que não sairás dali enquanto não pagares o derradeiro ceitil.
\end{quote}

De novo a Revista e Atualizada, que a gente tanto critica (e com
razão) traduzindo mais fielmente do que a \emph{gente}. Confiram o
grego do Texto Recebido: ``o derradeiro LEPTO''. Ver itens \ref{asse}, \ref{ceitil} e \ref{denario}.


\subsection*{Lc 14.10:}
\begin{quote}
    \small
Mas, quando fores convidado, vai, e assenta-te no
derradeiro lugar, para que, quando vier o que te convidou, te diga:
Amigo, \uwave{sobe mais para cima}. Então terás honra diante dos que estiverem
contigo à mesa.
\end{quote}

Mais um exemplo de pleonasmo vicioso: ``subir para cima'', ``descer
para baixo'', ``sair para fora'', ``entrar para dentro'', ``elo de
ligação'', etc. Ver \ref{saiu}.

King James: ``Friend, go up higher''.


\subsection*{Lc 14.26:}

\begin{quote}
    \small
Se alguém vier a mim, e não \uwave{aborrecer} a seu pai,
e mãe, e mulher, e filhos, e irmãos, e irmãs, e ainda também a sua própria
vida, não pode ser meu discípulo.
\end{quote}

Uma vez comentei essa minha dúvida com vários irmãos, que,
respeitosamente, na ocasião, não concordaram. Mas volto a citá-la
para a consideração da Sociedade Bíblica Trinitariana do Brasil.

O termo grego aqui é o ODIAR, tanto que a King James traduz por
\textit{to hate}. Interessante observar que, apesar do TC também traduzir por
``aborrecer'', a Bíblia de Jerusalém traduz por ``odiar'', com uma
nota: ``Hebraísmo: Jesus não exige ódio, mas desapego completo e
imediato (cf.~9.57-62)''.

Sou da opinião que traduzir por ``aborrecer'' arrefece, esfria e interpreta o significado do verbo original. Se Jesus falou ``odiar'' devemos procurar entender o que ele quis dizer com isso e não nos escandalizarmos com essa sua colocação em especial.

\subsection*{Lc 15.5:}
\begin{quote}
    \small
E achando-a, a põe sobre os seus ombros, \uwave{gostoso};\ldots
\end{quote}

``Gostoso''!? Vocês têm certeza de que essa seria a melhor palavra a
ser aplicada neste caso? Outra vez, acho infeliz essa tradução. A King
James registra: ``rejoicing'' (``rejoice''~=~alegrar-se, exultar) ,
que corresponde a ``regozijo, júbilo''. O TC traduz por ``alegre,
cheio de júbilo'' (BJ e RA, respectivamente).

\subsection*{Lc 21.2:}
\begin{quote}
    \small
E viu também uma pobre viúva lançar ali \uwave{duas pequenas moedas}; \ldots
\end{quote}

Sejamos mais literais: \texttt{dois leptos}. De novo a tradução do TC está mais exata do que a Almeida Corrigida e Fiel. Ver também os itens \ref{asse}, \ref{ceitil} e \ref{denario}.


\subsection*{Lc 22.62:}
\begin{quote}
    \small
E, \uwave{saindo Pedro para fora}, chorou amargamente.
\end{quote}

Pleonasmo vicioso. Ver item \ref{saiu}.

\section{João}
\subsection*{Jo 3.7:}
\begin{quote}
    \small
Não te maravilhes \uwave{de te ter dito}: Necessário vos é nascer de novo.
\end{quote}

Aliteração. É até difícil de se pronunciar. Poderia estar melhor
construído, como ``por eu te haver dito'' ou ``por eu ter dito a ti''.

King James: ``Marvel not that I said unto thee, Ye must be born again''.

RA: ``Não te admires de eu te haver dito: Necessário vos é nascer de novo''.

\subsection*{Jo 6.7:}
\begin{quote}
    \small
Filipe respondeu-lhe: Duzentos \uwave{dinheiros} de pão não
lhes bastarão, para que cada um deles tome um pouco.
\end{quote}

Ver item \ref{denario}.

\subsection*{Jo 6.17:}
\begin{quote}
    \small
E, entrando no barco, atravessaram o mar em direção a Cafarnaum; e era já escuro, e ainda Jesus não tinha chegado \uwave{ao pé deles}.
\end{quote}

Perguntaria, apenas, se no original está exatamente com essa expressão: ``ao pé deles''.

A King James registra: ``And entered into a ship, and went over the sea toward Capernaum. And it was now dark, and Jesus was not come to them''.

Apenas como exemplo cito: ``Não tinha vindo ter com eles'' (RA) e ``ainda não viera encontrá-los'' (BJ).

\subsection*{Jo 6.22-24:}
\begin{quote}
    \small
No dia seguinte, a multidão que estava do outro lado do mar, vendo que não havia ali mais do que um barquinho, a não ser aquele no qual os discípulos haviam entrado, e que Jesus não entrara com os seus discípulos naquele barquinho, mas que os seus discípulos tinham ido sozinhos (contudo, outros barquinhos tinham chegado de Tiberíades, perto do lugar onde comeram o pão, havendo o Senhor dado graças)\uwave{.} Vendo, pois, a multidão que Jesus não estava ali nem os seus discípulos, entraram eles também nos barcos, e foram a Cafarnaum, em busca de Jesus.''
\end{quote}

Passagem muito truncada. Poder-se-ia ter dado uma redação melhor.

Vejam a King James: ``The day following, when the people which stood on the other side of the sea saw that there was none other boat there, save that one whereinto his disciples were entered, and that Jesus went not with his disciples into the boat, but that his disciples were gone away alone; (howbeit there came other boats from Tiberias nigh unto the place where they did eat bread, after that the Lord had given thanks) when the people therefore saw that Jesus was not there, neither his disciples, they also took shipping, and came to Capernaum, seeking for Jesus.''

A Revista e Atualizada apresenta uma redação enxuta e muito melhor no
sentido da \texttt{compreensão textual}: ``No dia seguinte, a multidão que
ficara do outro lado do mar notou que ali não havia senão um pequeno
barco e que Jesus não embarcara nele com seus discípulos, tendo estes
partido sós. Entretanto, outros barquinhos chegaram de Tiberíades,
perto do lugar onde comeram o pão, tendo o Senhor dado graças. Quando,
pois, viu a multidão que Jesus não estava ali nem os seus discípulos,
tomaram os barcos e partiram para Cafarnaum à sua procura.'' Não é que
eu a recomende e a ache superior à da ACFiel, já que a RA parece
suprimir algumas palavras que, pelo que me consta, aparecem no texto
grego.

Portanto, nesta passagem, a King James apresenta todas as palavras do grego e com uma redação superior à da ACFiel publicada pela Sociedade Bíblica Trinitariana do Brasil.


\subsection*{Jo 11.28:}
\begin{quote}
    \small
E, dito isto, partiu, e chamou em segredo a Maria, sua irmã, dizendo: O Mestre \uwave{está cá}, e chama-te.
\end{quote}

``Está cá'', prezados editores da Sociedade Bíblica Trinitariana do
Brasil, é português de Portugal, não do Brasil. Por favor, mudem para
algo como: ``O Mestre está aqui, e te chama'', por exemplo.

\subsection*{Jo 11.38:}
\begin{quote}
    \small
Jesus, pois, \uwave{movendo-se outra vez muito em si mesmo}, veio ao sepulcro; e era uma caverna, e \uwave{tinha} uma pedra posta sobre ela.
\end{quote}

Perguntaria se no grego as palavras em destaque estão exatamente assim. ``Comovendo-se outra vez profundamente'' parece se aplicar melhor neste caso.

A  King James registra: ``Jesus therefore again groaning in himself cometh to the grave. It was a cave, and a stone lay upon it''.

Ainda: como a Bíblia se utiliza de uma linguagem culta, ``HAVIA uma pedra posta sobre ela'' cabe melhor do que ``TINHA uma pedra posta sobre ela''.

\subsection*{Jo 11.43:}
\begin{quote}
    \small
E, tendo dito isto, clamou com grande voz: Lázaro, \uwave{sai para fora}.
\end{quote}

Como comentei no item \ref{saiu}, este versículo não dá para ler em voz alta no púlpito das nossas igrejas, pois vamos cometer, digamos assim, um crime contra a boa gramática da Língua Portuguesa.

King James: ``And when he thus had spoken, he cried with a loud voice,
Lazarus, come forth.'' Ou seja, ``Lázaro, vem para fora!'', ou, simplesmente, ``saia!''


\subsection*{Jo 12.5:} 
\begin{quote}
    \small
Por que não se vendeu este ungüento por trezentos \uwave{dinheiros} e não se deu aos pobres?
\end{quote}

Mais uma vez, enfatizo: \texttt{denários}. Ver item \ref{denario}.


\subsection*{Jo 18.29:} 
\begin{quote}
    \small
Então Pilatos \uwave{saiu fora} e disse-lhes: Que acusação
trazeis contra este homem?
\end{quote}

Sem comentários (cf.~item \ref{saiu}).

\subsection*{Jo 19.4,5:}
\begin{quote}
    \small
Então Pilatos \uwave{saiu outra vez fora}, e disse-lhes: Eis aqui vo-lo trago fora, para que saibais que não acho nele crime algum. \uwave{Saiu, pois, Jesus fora}, levando a coroa de espinhos e roupa de púrpura. E disse-lhes Pilatos: Eis aqui o homem.
\end{quote}

Novamente, mais uma afronta à Gra\-má\-tica da Lín\-gua Por\-tu\-gue\-sa (i\-tem \ref{saiu}).

\chapter{Livros Históricos}
\section{Atos dos Apóstolos}

\subsection*{At 4.36:}
\begin{quote}
    \small
Então José, cognominado pelos apóstolos\uwave{,} Barnabé (que, traduzido, é Filho da consolação), levita, natural de Chipre,\ldots
\end{quote}

Não há essa vírgula após ``apóstolos''.


\subsection*{At 9.40:}
\begin{quote}
    \small
Mas Pedro, fazendo sair a todos, pôs-se de joelhos e orou\uwave{:} e, voltando-se para o corpo, disse: Tabita, levanta-te. E ela abriu os olhos, e, vendo a Pedro, assentou-se.
\end{quote}

Não há dois pontos após ``orou'', mas ponto-e-vírgula.


\subsection*{At 16.13:}
\begin{quote}
    \small
E no dia de sábado \uwave{saímos fora} das portas, para a beira do rio, onde se costumava fazer oração; e, assentando-nos, falamos às mulheres que ali se ajuntaram.
\end{quote}

De novo, o tal do pleonasmo vicioso (cf.~item \ref{saiu}).


\subsection*{At 23.24:}
\begin{quote}
    \small
E aparelhai cavalgaduras, para que, pondo nelas a Paulo, o levem salvo ao \uwave{presidente} Félix.
\end{quote}

Aqui, bem como nos versículos subsequentes (26, 33 e 34), eu
perguntaria se não seria mais correto o termo ``governador''. Ver \ref{govern}.

\subsection*{At 26.30:}
\begin{quote}
    \small
E, dizendo ele isto, levantou-se o rei, o \uwave{presidente}, e Berenice, e os que com eles estavam assentados.
\end{quote}
Idem acima.

\chapter{Cartas}
\section{Romanos}
\subsection*{Rm 5.12:}
\begin{quote}
    \small
Portanto, como por um homem entrou o pecado no mundo, e
 pelo pecado a morte, assim também a morte passou a todos os \uwave{homens
 por isso} que todos pecaram.
\end{quote}

Vamos colocar \emph{pelo menos} uma vírgula depois de ``homens'', não é mesmo?


\section{I Coríntios}
\subsection*{1Co 5.13:}
\begin{quote}
    \small
Mas Deus julga os que estão de fora. Tirai \uwave{pois} dentre vós a esse iníquo.
\end{quote}

``Pois'' deveria estar entre vírgulas.

\section{II Coríntios}
\subsection*{2 Co 3.17:}
\begin{quote}
    \small
Ora, o Senhor \uwave{é Espírito}; e onde está o Espírito do
Senhor, aí há liberdade.
\end{quote}

Não seria ``é O Espírito''? Sempre cismei com essa falta do artigo definido \emph{o}, até porque é ``Espírito'' em maiúsculo: pede artigo. Dando uma olhadela no grego, constatei: \textit{Ho de Kyrios einai TO Pneuma}. Ou seja, no grego tem o artigo. Qual seria a opinião dos irmãos neste caso?

\section{Efésios}
\subsection*{Ef 1.13-14:}
\begin{quote}
    \small
Em quem também vós estais, depois que  ouvistes a palavra da verdade, o evangelho da vossa salvação; e,  tendo nele também crido, fostes selados com o Espírito Santo da
promessa\uwave{.} O qual é o penhor da nossa  herança, para redenção da possessão adquirida, para louvor da sua  glória.
\end{quote}

Creio não ser ponto entre um versículo e outro, mas simplesmente uma vírgula.

\section{Filipenses}
\subsection*{Fp 3.5-6:}
\begin{quote}
    \small
Circuncidado ao oitavo dia, da linhagem de Israel, da tribo de Benjamim, hebreu de hebreus; segundo a lei, fui fariseu; segundo o zelo, perseguidor da  igreja, segundo a justiça que há na lei, irrepreensível.
\end{quote}

No versículo 6, manter a utilização do ponto-e-vírgula: ``\ldots{}; segundo o zelo, perseguidor da igreja; segundo a justiça que há na lei, irrepreensível.''

\section{II Tessalonicenses}
\subsection*{2Ts 2.11,12:}
\begin{quote}
    \small
E por isso Deus lhes enviará a
operação do erro, para que \uwave{creiam a mentira}; para que sejam julgados todos os que não \uwave{creram a verdade}, antes tiveram prazer na iniqüidade.
\end{quote}
Para que creiam ``a mentira''? Não, mas ``na mentira''. Igualmente: ``na verdade'' em vez de ``a verdade''.

\section{I Timóteo}
\subsection*{1Tm 1.2:}
\begin{quote}
    \small
A Timóte\uwave{o m}eu verdadeiro filho na fé: Graça,
misericórdia e paz da parte de Deus nosso Pai, e da de Cristo Jesus, nosso
Senhor.
\end{quote}

Falta uma vírgula depois de ``Timóteo''.

\section{Filemom}
\subsection*{Fm 10-16:}
\begin{quote}
    \small
Peço-te por meu filho Onésimo, que gerei nas minhas prisões; o qual noutro tempo te foi inútil, mas, agora, a ti e a mim, muito útil; eu to tornei a enviar. E \uwave{tu torna} a recebê-lo como às minhas entranhas. Eu bem o quisera conservar comigo, para que por ti me servisse nas prisões do evangelho; mas nada quis fazer sem o teu parecer, para que o teu benefício não fosse como por força, mas voluntário. Porque bem pode ser que ele se tenha separado de ti por algum tempo, para que o retivesses para sempre, não já como servo, antes, mais do que servo, como irmão amado, particularmente de mim, e quanto mais de ti, assim na carne como no Senhor\uwave{?}.
\end{quote}

Em primeiro lugar, no versículo 12, é ``tu tornas'' (a não ser que seja imperativo, aí seria ``torna tu'') e não ``tu torna'' --- faltou o `s', simplesmente. E, em segundo lugar, se os irmãos forem fazer uma leitura uma pouco mais atenta, vão se deparar com um ponto de interrogação ao final do versículo 16 e que, do jeito que está, não parece se tratar de uma pergunta. Na Revista e Corrigida também está
desse jeito e nas versões do TC não há o ponto de interrogação. Não é que não haja o ponto ou que não seja de fato uma pergunta, mas a maneira como foi traduzido o versículo 15 para o português não enseja uma pergunta.

A King James também enfatiza a pergunta, vejam: ``I beseech thee for my son Onesimus, whom I have begotten in my bonds: Which in time past was to thee unprofitable, but now profitable to thee and to me: Whom I have sent again: thou therefore receive him, that is, mine own bowels: Whom I would have retained with me, that in thy stead he might have ministered unto me in the bonds of the gospel: But without thy mind would I do nothing; that thy benefit should not be as it were of
necessity, but willingly. For perhaps he therefore departed for a season, that thou shouldest receive him for ever; not now as a servant, but above a servant, a brother beloved, specially to me, but how much more unto thee, both in the flesh, and in the Lord?''

Para vocês que dominam a língua inglesa, reparem que aqui faz sentido o ponto de interrogação ao final, regido pelo ``For perhaps''. Mas na tradução para o português (e não é que eu queira que se traduza do inglês para o português) não fica claro. Teria ficado, na minha modesta opinião, melhor redigido se, em vez de ``Porque bem pode ser que ele se tenha separado de ti \ldots{}'', algo do tipo ``Quem sabe se ele se separou de ti \ldots{}''. Gostaria de saber as opiniões dos irmãos sobre esta passagem.

\section{Hebreus}
\subsection*{Hb 3.7-9:}
\begin{quote}
    \small
Portanto, como diz o Espírito Santo: Se ouvirdes hoje a sua voz, não endureçais os vossos corações, como na provocação, no dia da tentação no deserto\uwave{.} Onde vossos pais me tentaram, me provaram, e viram por quarenta anos as minhas obras.
\end{quote}

Não há ponto depois de ``deserto'', mas vírgula.

\section{Tiago}
\subsection*{Tg 5.4:}
\begin{quote}
    \small
Eis que o \uwave{jornal} dos trabalhadores que ceifaram as vossas terras, e que por vós foi diminuído, clama; e os clamores dos que ceifaram entraram nos ouvidos do Senhor dos exércitos.
\end{quote}

Jornal?! Termo fora de uso! Atualizem: \texttt{salário}.

\section{I João}
\subsection*{1Jo 2.4:}
\begin{quote}
    \small
Aquele que diz: Eu \uwave{conheço-o}, e não guarda os seus mandamentos, é mentiroso, e nele não está a verdade.
\end{quote}

Não está errado ``Eu conheço-o'', mas como o pronome ``Eu'' inicia a oração, recomenda-se ``Eu o conheço''.

\subsection*{1Jo 5.10:}
\begin{quote}
    \small
Quem crê no Filho de Deus, em si mesmo tem o
testemunho; quem a Deus não \uwave{crê mentiroso} o fez, porquanto não creu no testemunho que Deus de seu Filho deu.
\end{quote}

Falta uma vírgula após a palavra ``crê''.

\chapter{Revelação de Jesus Cristo}
\section{Apocalipse}
\subsection*{Ap 2.2:}
\begin{quote}
    \small
Conheço as tuas obras, e o teu trabalho, e a tua
 paciência, e que não podes \uwave{sofrer os maus}; e puseste à prova
 os que dizem ser apóstolos, e o não são, e tu os achaste
 mentirosos.
\end{quote}

``Suportar os maus'' cabe melhor aqui. A King James assina em baixo: ``\ldots{}and how thou canst not \emph{bear} them which are evil \ldots''

\subsection*{Ap 4.2:}
\begin{quote}
    \small
E logo fui arrebatado no \uwave{Espírito}, e eis que um trono estava posto no céu, e um assentado sobre o trono.
\end{quote}

Não seria ``espírito'' (em minúsculas)?

As demais versões concordam neste ponto:
\begin{itemize}
\item RA: Imediatamente, eu me achei em espírito, e eis armado no céu um
trono, e, no trono, alguém sentado;\ldots
\item RC: E logo fui arrebatado em espírito, e eis que um trono estava
posto no céu, e um assentado sobre o trono.
\item King James: And immediately I was in the spirit: and, behold, a throne was set in heaven, and one sat on the throne.
\end{itemize}

\subsection*{Ap 6.5:}
\begin{quote}
    \small
E, havendo aberto o terceiro selo, \uwave{ouvi dizer ao terceiro animal}: Vem, e vê. E olhei, e eis um cavalo preto e o que sobre ele estava assentado tinha uma balança na mão.
\end{quote}

Minha dúvida: Quem disse e a quem? Não seria ``ouvi dizer \emph{o} terceiro animal''? Foi o terceiro animal que disse e não alguém disse \emph{ao} terceiro animal.

\begin{itemize}
\item King James: And when he had opened the third seal, I heard the third beast say, Come and see. And I beheld, an lo a black horse; and he that sat on him had a pair of balances in his hand.
\end{itemize}

\subsection*{Ap 6.6:}
\begin{quote}
    \small
E ouvi uma voz no meio dos quatro animais, que dizia: Uma
medida de trigo por um \uwave{dinheiro}, e três medidas de cevada por um \uwave{dinheiro}; e não danifiques o azeite e o vinho.
\end{quote}

Como já explicado (item \ref{denario}), a melhor tradução é ``denário'', não ``dinheiro''.

\subsection*{Ap 19.1:}
\begin{quote}
    \small
\uwave{E, depois destas coisas} ouvi no céu como que uma grande voz de uma grande multidão, que dizia: Aleluia! Salvação, e glória, e honra, e poder pertencem ao Senhor nosso Deus;\ldots
\end{quote}

Falta uma vírgula após: ``Depois destas coisas''.


\subsection*{Ap 22.12:}
\begin{quote}
    \small
E\uwave{,} eis que cedo venho, e o meu galardão está comigo, para dar a cada um segundo a sua obra.
\end{quote}

Não há essa primeira vírgula, após ``E''.
